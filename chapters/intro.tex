\chapter{\ifenglish Introduction\else บทนำ\fi}

% นิยาม environment mypara สำหรับย่อหน้าใหม่หลัง section
\newenvironment{mypara}{\par\indent}{}

% กำหนดให้ย่อหน้าอัตโนมัติ
\setlength{\parindent}{2em}

\section{\ifenglish Project rationale\else ที่มาของโครงงาน\fi}
\begin{mypara}
    \indent การออกแบบระบบขนส่งสาธารณะโดยรถบัสเป็นงานที่ซับซ้อน เนื่องจากต้องคำนึงถึงความหลากหลายของโครงข่ายเส้นทาง 
    สภาพแวดล้อมทางกายภาพ ความหนาแน่นของประชากร \\ 
    \indent โดยเฉพาะในบริบทของมหาวิทยาลัยเชียงใหม่ ระบบขนส่งสาธารณะมีความซับซ้อนเป็นพิเศษ 
    เนื่องจากต้องให้บริการครอบคลุมทั้ง คณะ อาคารเรียน และหอพักนักศึกษา โดยมีจุดรับ-ส่งมากถึง 56 ป้าย ครอบคลุม 9 เส้นทางหลัก 
    ขณะที่นักศึกษามีการเปลี่ยนอาคารเรียนตามช่วงเวลา การใช้รถบัสไฟฟ้ายังเพิ่มข้อจำกัดด้านระยะทางและเวลาวิ่งต่อรอบ 
    ส่งผลให้การกำหนดเส้นทางและความถี่ในการให้บริการมีความท้าทายยิ่งขึ้น \\
    \indent ด้วยความซับซ้อนดังกล่าว การใช้แบบจำลองจึงเป็นเครื่องมือสำคัญในการออกแบบและวิเคราะห์ระบบขนส่ง 
    โดยช่วยจำลองเส้นทางที่หลากหลาย จำลองความถี่ของการออกรถ วิเคราะห์ข้อมูลที่จำเป็น เพื่อช่วยให้ผู้วางแผนสามารถตัดสินใจได้อย่าง
    แม่นยำและมีประสิทธิภาพมากขึ้น  \\
    \indent โปรแกรมจำลอง ในปัจจุบันส่วนใหญ่ไม่ได้ถูกออกแบบมาเพื่อระบบขนส่งสาธารณะโดยเฉพาะ \\
    แม้ว่าจะสามารถนำไปประยุกต์ใช้กับระบบขนส่งสาธารณะได้ในบางกรณี แต่การใช้งานจริงกลับค่อนข้างยุ่งยากและซับซ้อน 
    เนื่องจากโปรแกรมเหล่านั้นไม่ได้ถูกพัฒนาขึ้นเพื่อรองรับความต้องการเฉพาะของงานนี้ตั้งแต่ต้น โดยเฉพาะอย่างยิ่งกับระบบขนส่งสาธารณะ
    ประเภทรถบัส ซึ่งมีความซับซ้อนในเรื่องของเส้นทาง เวลาเดินรถ และความจุผู้โดยสารที่แตกต่างจากระบบขนส่งรูปแบบอื่นๆ 
    จึงทำให้โปรแกรมจำลอง ในปัจจุบันยังไม่สามารถตอบโจทย์การวางแผนและบริหารจัดการระบบขนส่งรถบัสได้อย่างมีประสิทธิภาพ
\end{mypara}

\section{\ifenglish Objectives\else วัตถุประสงค์ของโครงงาน\fi}
\begin{enumerate}
    \item เพื่อออกแบบและพัฒนาระบบจำลองระบบขนส่งสาธารณะประเภทรถบัสที่สามารถวิเคราะห์ข้อมูลที่เกี่ยวข้อง เช่น อัตราการใช้งานของรถบัส เวลารอโดยเฉลี่ยของผู้โดยสาร และอัตราการเกิดเหตุการณ์ที่รถบัสมีการใช้งานเกินเวลาหรือระยะทางที่กำหนด
    \item เพื่อให้ระบบจำลองที่พัฒนาขึ้นมีความแม่นยำและถูกต้องตามสถิติและสอดคล้องกับเหตุการณ์ที่เกิดขึ้นจริง 
    \item เพื่อออกแบบและพัฒนาปรับปรุงระบบให้สอดคล้องกับความต้องการของผู้ใช้งานจริง
\end{enumerate}

\section{\ifenglish Project scope\else ขอบเขตของโครงงาน\fi}

\subsection{\ifenglish Software scope\else ขอบเขตด้านซอฟต์แวร์\fi}
    \begin{mypara}
        \indent ระบบถูกออกแบบเพื่อใช้จำลองข้อมูลจากตารางการเดินรถ โดยอาศัยข้อมูลสถิติในอดีตเป็นหลัก 
        ไม่ได้อ้างอิงจากข้อมูลแบบ Real-time ทั้งนี้ระบบสามารถประยุกต์ใช้งานได้กับระบบรถ 
        Shuttle Bus โดยทั่วไป ไม่จำกัดว่าต้องพื้นที่ใดพื้นที่นึงเท่านั้น
    \end{mypara}
\subsection{\ifenglish Hardware scope\else ขอบเขตด้านฮาร์ดแวร์\fi}
    \begin{mypara}
        \indent ผู้ใช้งานสามารถเข้าถึงระบบผ่านเครื่องคอมพิวเตอร์ส่วนบุคคล (PC หรือ Laptop)
         โดยยังไม่รองรับการใช้งานผ่านอุปกรณ์พกพา (Mobile Device)
    \end{mypara}
\subsection{\ifenglish Data scope\else ขอบเขตด้านข้อมูล\fi}
    \begin{mypara}
        \indent ระบบใช้ข้อมูลจากตารางการเดินรถและข้อมูลเชิงสถิติย้อนหลังมาเป็นฐานในการวิเคราะห์และจำลองสถานการณ์ 
        โดยในกรณีศึกษานี้จะมุ่งเน้นที่ระบบขนส่งมวลชนภายในมหาวิทยาลัยเชียงใหม่ จำนวน 2 สายบริการ
    \end{mypara}
\subsection{\ifenglish User scope\else ขอบเขตด้านผู้ใช้งาน\fi}
    \begin{mypara}
        \indent ระบบสามารถรองรับการใช้งานพร้อมกันได้หลายบุคคล (Multi-user) 
        กลุ่มผู้ใช้งานหลักประกอบด้วยผู้ดูแลระบบ ผู้วางแผนการเดินรถ และผู้บริหารที่เกี่ยวข้อง
    \end{mypara}

\section{\ifenglish Project constraints\else ข้อจำกัดของโครงงาน\fi}

\subsection{\ifenglish Time constraint\else ข้อจำกัดด้านเวลา\fi}
    \begin{mypara}
        \indent ข้อมูลบางประเภทได้มาจากการเก็บรวบรวมจากผู้ใช้จริง ซึ่งอาจทำให้เกิดช่วงเวลาที่ไม่สะดวกต่อการดำเนินการเก็บข้อมูล 
        นอกจากนี้ ระยะเวลาในการดำเนินโครงงานมีเพียง 1 ปี จึงอาจทำให้รายละเอียดบางประการของซอฟต์แวร์เกิดข้อผิดพลาดหรือคลาดเคลื่อนได้
    \end{mypara}
\subsection{\ifenglish Equipment constraints\else ข้อจำกัดด้านอุปกรณ์\fi}
    \begin{mypara}
        \indent จำนวนรถที่ให้บริการภายในมหาวิทยาลัยเชียงใหม่มีบางส่วนที่อยู่ในสภาพชำรุด 
        ส่งผลให้จำนวนรถที่ใช้งานจริงอาจแตกต่างจากข้อมูลที่นำมาใช้ในการจำลอง (Simulation) 
        ซึ่งอาจก่อให้เกิดความคลาดเคลื่อนของผลลัพธ์ นอกจากนี้ ยังมีโอกาสที่รถบางคันออกให้บริการไม่เป็นไปตามตารางเวลา 
        ส่งผลให้ผลลัพธ์จากการจำลองอาจไม่ตรงกับสภาพความเป็นจริงเช่นกัน
    \end{mypara}

\section{\ifenglish Expected outcomes\else ประโยชน์ที่ได้รับ\fi}
    \begin{mypara}
        \indent ระบบจำลองจะช่วยสนับสนุนการตัดสินใจในการนำตารางการเดินรถไปประยุกต์ใช้จริงได้อย่างมีประสิทธิภาพและเหมาะสมมากยิ่งขึ้น
    \end{mypara}
\section{\ifenglish Technology and tools\else เทคโนโลยีและเครื่องมือที่ใช้\fi}

\subsection{\ifenglish Software technology\else เทคโนโลยีด้านซอฟต์แวร์\fi}
    \begin{mypara}
        \indent โครงงานนี้ใช้เทคโนโลยีด้านซอฟต์แวร์และเครื่องมือดังนี้ ส่วนของ Display Frontend ใช้ Vite และ TypeScript 
        สำหรับการพัฒนา UI/UX และการโต้ตอบ พร้อมทั้งใช้ Leaflet.js สำหรับการแสดงผลแผนที่ ส่วนของ Logical Model 
        ใช้ Python ร่วมกับ FastAPI โดยอาศัย Salabim ในการสร้างและควบคุมกระบวนการจำลอง และใช้ SciPy (fitcurve) 
        สำหรับการวิเคราะห์เชิงคณิตศาสตร์ ในส่วนต่อมา API Controller ใช้ Go fiber สำหรับการจัดการการยืนยันตัวตน 
        การเชื่อมต่อและบริหารจัดการฐานข้อมูล การกำหนดเส้นทาง API และการจัดการคำร้องขอ 
        ส่วน Database ใช้ PostgreSQL เพื่อการจัดเก็บและบริหารข้อมูล นอกจากนี้ ในส่วนของ 
        Infrastructure ใช้ Docker สำหรับสร้างสภาพแวดล้อมที่สม่ำเสมอและง่ายต่อการปรับใช้งาน และใช้ DigitalOcean 
        สำหรับโฮสต์ระบบและให้บริการออนไลน์
    \end{mypara}


\section{\ifenglish Project plan\else แผนการดำเนินงาน\fi}

\begin{plan}{6}{2025}{2}{2026}
    \planitem{6}{2025}{6}{2025}{ระบุและนิยามปัญหา}
    \planitem{7}{2025}{7}{2025}{ระบุ Stakeholder}
    \planitem{8}{2025}{8}{2025}{User Study}
    \planitem{8}{2025}{10}{2025}{ศึกษาหลักการที่เกี่ยวข้อง}
    \planitem{8}{2025}{10}{2025}{กำหนดวิธีการแก้ปัญหา}
    \planitem{9}{2025}{10}{2025}{การออกแบบ Userflow และ Wireframe}
    \planitem{10}{2025}{11}{2025}{ออกแบบ UX/UI}
    \planitem{11}{2025}{1}{2026}{พัฒนา Prototype แรก}
    \planitem{11}{2025}{1}{2026}{การเก็บรวบรวมข้อมูลสำหรับการทดสอบแบบจำลอง}
    \planitem{1}{2026}{2}{2026}{การทดสอบและการเก็บ Feedback จากผู้ใช้งานจริง}
    \planitem{1}{2026}{2}{2026}{พัฒนาและปรับปรุงซอฟแวร์}
\end{plan}

\section{\ifenglish Roles and responsibilities\else บทบาทและความรับผิดชอบ\fi}
\begin{mypara}
    \indent ในการทำโครงงานนี้สมาชิกแต่ละคนมีบทบาทและหน้าที่ที่รับผิดชอบดังนี้
\end{mypara}
\begin{enumerate}
    \item \textbf{นางสาวปวีณนุช เพราะสุนทร รหัส 650610783} มีบทบาทเป็น Project Owner และ Developer 
    โดยรับผิดชอบการกำหนดขอบเขตงานให้ตรงความต้องการของระบบ ออกแบบและพัฒนา Logical Model ที่เป็นส่วนของ Backend 
    และรับผิดชอบการพัฒนาและดูแล API Controller เพื่อเชื่อมต่อระหว่าง Frontend และ Backend
    \item \textbf{นายอนุวัตร เอี่ยมศรี รหัส 650610819} มีบทบาทเป็น Developer โดยรับผิดชอบการพัฒนาส่วน Display Frontend เพื่อให้ผู้ใช้สามารถโต้ตอบกับระบบได้ และดูแลการออกแบบและจัดการ 
    ฐานข้อมูล (Database) ให้มีโครงสร้างที่ถูกต้อง และสอดคล้องกับความต้องการของระบบ
    \item \textbf{นางสาวอิศวรา คงศรีเจริญ รหัส 650610821} มีบทบาทเป็น Developer และ Designer โดยรับผิดชอบการออกแบบและวางระบบ Infrastructure เพื่อสนับสนุนการทำงานภายในของระบบ 
    และออกแบบ UX/UI Design ให้เหมาะสมกับการใช้งานระบบ
\end{enumerate}
\section{\ifenglish%
Impacts of this project on society, health, safety, legal, and cultural issues
\else%
ผลกระทบด้านสังคม สุขภาพ ความปลอดภัย กฎหมาย และวัฒนธรรม
\fi}
\begin{mypara}
    \indent ด้านสังคม โครงงานนี้มีส่วนช่วยให้ผู้ใช้และผู้วางแผนนโยบายเข้าใจรูปแบบการเดินทางและพฤติกรรมการใช้ระบบขนส่งสาธารณะในชุมชนได้อย่างลึกซึ้งขึ้น โดยการวิเคราะห์ข้อมูลการเดินทางจริงร่วมกับการจำลองสถานการณ์ สามารถเห็นแนวโน้ม 
    เช่น ช่วงเวลาที่ผู้โดยสารหนาแน่น เส้นทางที่มีความต้องการสูง หรือพฤติกรรมการต่อคิวและขึ้นลงรถ ซึ่งข้อมูลเหล่านี้สามารถนำไปใช้ในการวางแผนและปรับปรุงเส้นทางรถ การจัดสรรรถและเวลาการเดินรถให้เหมาะสมกับความต้องการของผู้โดยสาร 
    ส่งผลให้ระบบขนส่งมีประสิทธิภาพมากขึ้น นอกจากนี้ยังช่วยยกระดับคุณภาพชีวิตของผู้คน เนื่องจากผู้โดยสารจะได้รับบริการที่สะดวก ปลอดภัย และลดเวลารอคอย อีกทั้งยังสนับสนุนการตัดสินใจด้านสังคม เช่น 
    การพัฒนาพื้นที่โดยรอบสถานีขนส่งและการกำหนดนโยบายส่งเสริมการใช้ขนส่งสาธารณะ

    \indent ด้านความปลอดภัย การใช้การจำลอง เป็นเครื่องมือสำคัญในการประเมินความเสี่ยงและทดสอบสถานการณ์ต่าง ๆ โดยไม่ต้องเสี่ยงต่อผู้ใช้งานจริง ตัวอย่างเช่น การตรวจสอบว่าการออกรอบแต่ละครั้ง 
    รถโดยสารสามารถวิ่งภายใต้ลิมิตเวลาที่กำหนดหรือไม่ หากเกินลิมิต อาจทำให้ระบบรถทำงานหนักจนดับ หรือเพิ่มความเสี่ยงต่อการเกิดอุบัติเหตุ การจำลองสถานการณ์เหล่านี้ช่วยให้สามารถวางมาตรการป้องกันล่วงหน้า 
    เช่น การปรับแผนรอบเดินรถ หรือการจัดสรรทรัพยากรให้เหมาะสม ส่งผลให้ระบบขนส่งสาธารณะมีความปลอดภัยและเชื่อถือได้มากยิ่งขึ้น

    \indent การปรับใช้เชิงวิศวกรรม การใช้การจำลองในเชิงวิศวกรรมช่วยให้สามารถออกแบบและปรับปรุงระบบขนส่งสาธารณะอย่างเป็นระบบและมีหลักวิศวกรรมรองรับ โดยสามารถนำข้อมูลจากการจำลองมาใช้ในการวิเคราะห์ปัญหาและออกแบบโครงสร้างการเดินรถ 
    และการบริหารจัดการทรัพยากรระบบรถ เพื่อให้รถทำงานได้อย่างปลอดภัยและมีประสิทธิภาพนอกจากนี้ การจำลองยังช่วยในการทดสอบสมมติฐานต่าง ๆ ก่อนนำไปใช้งานจริง ทำให้วิศวกรสามารถออกแบบมาตรการป้องกัน ปรับปรุงระบบ และวางแผนบำรุงรักษาได้อย่างเหมาะสม
\end{mypara}

\section{\ifenglish Project budget plan\else แผนการใช้งบประมาณ\fi}
\begin{enumerate}
    \item ค่าสมาชิกรายเดือนของ Canva Pro สำหรับการออกแบบเชิงกราฟิก\\
    \textbf{งบประมาณ:} 230 บาท/เดือน จำนวน 1 บัญชี เป็นเวลา 10 เดือน รวมเป็นเงิน 2300 บาท
    \item ค่าสมาชิกรายเดือนของ Overleaf สำหรับการเขียนเอกสาร \LaTeX \\
    \textbf{งบประมาณ:} 300 บาท/เดือน จำนวน 1 บัญชี เป็นเวลา 5 เดือน รวมเป็นเงิน 1500 บาท
    \item ค่าเช่ารายเดือนของ Droplet ผ่าน DigitalOcean สำหรับการพัฒนาและโฮสติ้งของระบบ\\
    \textbf{งบประมาณ:} 387 บาท/เดือน (2 GB / 1 CPU, 50 GB SSD Disk, 2 TB transfer) \\
    จำนวน 1 เครื่อง เป็นเวลา 5 เดือน รวมเป็นเงิน 1935 บาท
    \item ค่าเช่ารายเดือนของ Additional Storage ของ Droplet ผ่าน DigitalOcean สำหรับการ\\
    เก็บข้อมูล Workspace\\
    \textbf{งบประมาณ:} 162 บาท/เดือน ขนาด 50 GB เป็นเวลา 5 เดือน รวมเป็นเงิน 810 บาท
    \item ค่าโดเมนเนม .com ผ่าน Godaddy สำหรับการเข้าถึงระบบผ่านทาง URL ที่จดจำง่าย\\
    \textbf{งบประมาณ:} 400 บาท/ปี จำนวน 1 โดเมน รวมเป็นเงิน 400 บาท
    \item ค่าใช้จ่ายสำหรับการรวบรวมข้อมูลในการทดสอบระบบ\\
    \textbf{งบประมาณ:} 1500 บาท ตลอดการดำเนินโครงงาน
    \item ค่าอุปกรณ์สำนักงาน และการจัดทำโปสเตอร์\\
    \textbf{งบประมาณ:} รวมเป็นเงิน 1000 บาท
    \item ค่าใช้จ่ายสำหรับการเดินทาง(สมาชิกที่ไม่มีรถส่วนตัว)\\
    \textbf{งบประมาณ:} 600 บาท จำนวน 3 คน ตลอดการดำเนินโครงงาน รวมเป็นเงิน 1800 บาท\\ \\
    \textbf{งบประมาณรวมทั้งสิ้น:} 11255 บาท

\end{enumerate}