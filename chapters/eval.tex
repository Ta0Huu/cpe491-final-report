\chapter{\ifproject%
\ifenglish Experimentation and Results\else การทดลองและผลลัพธ์\fi
\else%
\ifenglish System Evaluation\else การประเมินระบบ\fi
\fi}
\begin{mypara}
    \indent บทนี้จะกล่าวถึงการประเมินระบบทั้งในด้านการทำงานของระบบ ความถูกต้องและความแม่นยำของระบบจำลอง 
    รวมถึงการประเมินจากผู้ใช้งานจริง เพื่อยืนยันความเหมาะสมและประสิทธิภาพของการนำระบบไปประยุกต์ใช้ในสถานการณ์จริง
    โดยมีรายละเอียดการประเมินดังนี้
\end{mypara}

\section{การวัดและประเมินการทำงานของระบบจำลอง}
\begin{itemize}
    \item \textbf{ความถูกต้องในการ Login และ Sign up}
        \\ ประเมินว่าระบบสามารถให้ผู้ใช้ลงทะเบียนและเข้าสู่ระบบได้อย่างถูกต้องหรือไม่
    \item \textbf{ความถูกต้องในการจัดเก็บข้อมูลใน Workspace}
        \\ ประเมินว่าระบบสามารถจัดเก็บข้อมูล Configuration data และ Input ใน Workspace ของผู้ใช้ได้อย่างถูกต้องหรือไม่
    \item \textbf{ความถูกต้องในการเก็บข้อมูล ของผู้ใช้ที่ลงทะเบียน}
        \\ ประเมินว่าระบบสามารถเก็บข้อมูล Work และ Configuration data ได้อย่างถูกต้องหรือไม่
    \item \textbf{ความถูกต้องในการดึงข้อมูล Map data จาก OpenStreetMap}
        \\ ประเมินว่าระบบสามารถดึงข้อมูลแผนที่จาก OpenStreetMap ได้ถูกต้องและครบถ้วนตามพื้นที่ที่ผู้ใช้เลือกหรือไม่
    \item \textbf{ความถูกต้องในการสร้าง Configuration data}
        \\ ประเมินว่าระบบบสามารถรับข้อมูลที่ผู้ใช้กรอกมาใน Configuration form และสร้าง Configuration data ที่ถูกต้องตามรูปแบบที่กำหนดได้หรือไม่
    \item \textbf{ความถูกต้องในการรับข้อมูล input data}
        \\ ประเมินว่าระบบสามารถรับข้อมูล Input data ที่ผู้ใช้กรอกมาใน Input form และสร้าง Input data ที่ถูกต้องและนำไปใช้รูปแบบที่กำหนดได้หรือไม่
    \item \textbf{ความถูกต้องในการทำงานของระบบจำลอง}
        \\ ประเมินว่าการทำงานของระบบจำลองสามารถทำงานได้ถูกต้องตามที่ออกแบบไว้หรือไม่
    \item \textbf{ความถูกต้องในการแสดงผลลัพธ์}
        \\ ประเมินว่าระบบสามารถแสดงผลลัพธ์ที่ได้จากการจำลองในรูปแบบสถิติและกราฟได้อย่างถูกต้องหรือไม่

\end{itemize}
\section{การวัดและประเมินความถูกต้องและความแม่นยำของระบบจำลอง}
\begin{mypara}
    \indent การประเมินความถูกต้องและความแม่นยำของระบบจำลองจะทำได้โดยการเปรียบเทียบผลลัพธ์ที่ได้จาก
    การจำลองกับข้อมูลจริงโดยจะมีการนำข้อมูลจริงที่ได้จากการสำรวจจริงในช่วงเวลาเดียวกับและสายรถเดียวกัน มาเปรียบเทียบผลลัพธ์เพื่อตรวจสอบความแม่นยำ
\end{mypara}
\section{การวัดและประเมินด้านประสิทธิภาพการใช้งานจากผู้ใช้จริง}
\begin{mypara}
    \indent การประเมินด้านประสิทธิภาพการใช้งานจากผู้ใช้จริงจะทำได้โดยการให้ผู้ใช้ทดลองใช้งานระบบจำลอง
    และทำแบบสอบถามเพื่อเก็บข้อมูลความคิดเห็นและข้อเสนอแนะเกี่ยวกับการใช้งานระบบในด้านต่าง ๆ เช่น
    ความง่ายในการใช้งาน ความรวดเร็วในการประมวลผล ความพึงพอใจในการใช้งาน เป็นต้น
\end{mypara}