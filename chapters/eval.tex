\chapter{\ifproject%
\ifenglish Experimentation and Results\else การทดลองและผลลัพธ์\fi
\else%
\ifenglish System Evaluation\else การประเมินระบบ\fi
\fi}
\begin{mypara}
    \indent ในโครงงานนี้จะมีการทดสอบการทำงานของระบบจากข้อมูลจริงซึ่งเก็บรวบรวมจากการสำรวจ
    ของผู้โดยสารรถขนส่งสาธารณะของมหาวิทยาลัยเชียงใหม่ เป็นจำนวนอย่างน้อย 2 สายบริการ 
    โดยจะนำชุดข้อมูลมาเป็นข้อมูลการตั้งค่าเพื่อใช้ในการจำลองสถานการณ์ โดยจะมีการประเมินระบบ 
    ความถูกต้องและความแม่นยำของระบบจำลอง และมีการประเมินจากผู้ใช้งานจริง 
    โดยมีรายละเอียดการประเมินดังนี้
\end{mypara}


\section{การวัดและประเมินความถูกต้องและความแม่นยำของระบบจำลอง}
\begin{mypara}
    \indent การประเมินความถูกต้องและความแม่นยำของระบบจำลอง จะแบ่งการประเมินออกเป็น 2 ส่วน คือ
\end{mypara}
\subsection{การประเมินความถูกต้อง}
\begin{mypara}
    \indent การประเมินความถูกต้องของระบบจำลอง จะตรวจสอบว่าระบบสามารถทำงานได้ถูกต้องตามที่ออกแบบไว้หรือไม่
    ซึ่งแบ่งออกเป็น 2 ส่วนที่จะประเมิน ได้แก่
\end{mypara}
    \begin{enumerate}
        \item การประเมินความถูกต้องของระบบจำลอง
            \begin{mypara}
                \indent โดยจะตรวจสอบว่าระบบสามารถทำงานได้ถูกต้องตามที่ออกแบบไว้หรือไม่ โดยจะมีการทดสอบในแต่ละฟังก์ชันการทำงานของระบบ
                ดังนี้
            \end{mypara}
                \begin{itemize}
                    \item \textbf{ความถูกต้องในการ Login และ Sign up}
                        \\ ประเมินว่าระบบสามารถให้ผู้ใช้ลงทะเบียนและเข้าสู่ระบบได้อย่างถูกบัญชีหรือไม่
                    \item \textbf{ความถูกต้องในการจัดเก็บข้อมูลใน Workspace}
                        \\ ประเมินว่าระบบสามารถจัดเก็บข้อมูล Configuration data และ Scenario data ที่ผู้ใช้บันทึกใน Workspace ได้อย่างถูกต้องหรือไม่ 
                    \item \textbf{ความถูกต้องในการดึงข้อมูล Map data จาก OpenStreetMap}
                        \\ ประเมินว่าระบบสามารถดึงข้อมูลสถานีจากแผนที่ OpenStreetMap ได้ถูกต้องและครบถ้วนตามพื้นที่ที่ผู้ใช้เลือกหรือไม่
                    \item \textbf{ความถูกต้องในการทำงานของระบบจำลอง}
                        \\ ประเมินว่าการทำงานของระบบจำลองสามารถทำงานได้ถูกต้องโดยข้อมูลที่รับเข้ามา
                        ถูกนำเข้ากระบวนการจำลองและได้ผลลัพธ์จากการคำนวณข้อมูลนั้นเป็นผลลัพธ์
                \end{itemize}
        \item การประเมินความถูกต้องของผลลัพธ์เชิงสถิติ
            \begin{mypara}
                \indent จากการจำลองด้วยชุดข้อมูลทดสอบจำนวนอย่างน้อย 2 สายบริการ 
                โดยจะนำผลลัพธ์มาเปรียบเทียบกับข้อมูลจริงที่ได้กันจากการเก็บข้อมูลในช่วงเวลาเดียว
                และสายรถเดียวกัน ในการจำลองเพื่อดูความแตกต่างของผลลัพธ์ที่ได้จากการจำลองกับข้อมูลจริง \\
                \indent การประเมินผลจะวิเคราะห์ผลลัพธ์ทางสถิติว่ามีความสอดคล้องกับข้อมูลที่เกิดขึ้นจริงหรือไม่
            \end{mypara}
        \
    \end{enumerate}
\subsection{การประเมินความแม่นยำ}
\begin{mypara}
    \indent การประเมิณความแม่นยำของระบบจำลอง จะทำโดยทดลองระบบจำลองด้วยชุดข้อมูลทดสอบ
    เดียวกันอย่างน้อย 5 ครั้ง เพื่อดูความแตกต่างของผลลัพธ์ที่ได้จากการจำลองแต่ละครั้ง
\end{mypara}
\begin{enumerate}
    \item การประเมินความแม่นยำของผลลัพธ์เชิงสถิติ
            \begin{mypara}
                \indent จากการเปรียบเทียบผลลัพธ์ที่ได้จากการจำลองของข้อมูลแต่ละครั้งซึ่งใช้ข้อมูลชุดเดียวกัน 
                จะสามารถตรวจสอบความสม่ำเสมอและความเกาะตัวของผลลัพธ์ได้ โดยการประเมินนี้ช่วยให้เห็นว่า 
                การจำลองแต่ละครั้งให้ค่าผลลัพธ์ที่ใกล้เคียงกันหรือไม่ ซึ่งเป็นตัวชี้วัดความเสถียรของโมเดล 
            \end{mypara}
\end{enumerate}
\section{การประเมินความพึงพอใจและรับข้อเสนอแนะการใช้งานจากผู้ใช้จริง}
\begin{mypara}
    \indent การประเมินความพึงพอใจและรับข้อเสนอแนะการใช้งานจากผู้ใช้จริงจะทำโดยการให้ผู้ใช้งาน
    อย่างน้อย 5 คน ทดลองใช้งานระบบจำลอง และทำแบบสอบถามเพื่อเก็บข้อมูลความคิดเห็นและข้อเสนอแนะ
    โดยจะแบ่งออกเป็น 2 ส่วน คือ
    \subsection{การประเมินความพึงพอใจในการใช้งานระบบจำลอง}

        \begin{mypara}
            \indent โดยจะให้ผู้ใช้งานทำแบบสอบถามเพื่อประเมินความพึงพอใจในการใช้งานระบบจำลอง
            โดยใช้มาตราส่วนประมาณค่า (Rating Scale) แบบ 5 ระดับ ได้แก่
            \begin{itemize}
                \item 5 = มากที่สุด (Very Satisfied)
                \item 4 = มาก (Satisfied)
                \item 3 = ปานกลาง (Neutral)
                \item 2 = น้อย (Dissatisfied)
                \item 1 = น้อยที่สุด (Very Dissatisfied)
            \end{itemize}
            ซึ่งจะมีการประเมินในหัวข้อต่างๆ ดังนี้
            \begin{itemize}
                \item ความง่ายในการใช้งานระบบโดยรวม
                \item ความเข้าใจและอ่านง่ายของการแสดงผล
                \item ความสะดวกในการกรอกข้อมูลเพื่อใช้ในการจำลอง
                \item ความรวดเร็วในการประมวลผลข้อมูล
                \item ความพึงพอใจโดยรวมในการใช้งานระบบจำลอง
            \end{itemize}
        \end{mypara}

    \subsection{การรับข้อเสนอแนะการใช้งานระบบจำลอง}
        \begin{mypara}
            \indent โดยจะมีให้ผู้ใช้งานทำแบบสอบถามการทำงานของใช้งานได้อย่างถูกต้องหรือไม่
            พร้อมให้เสนอแนะของการปรับปรุงของแต่ละส่วนในระบบจำลองโดยจะมีหัวข้อในการสอบถาม ดังนี้
            \begin{itemize}
                \item การเข้าสู่ระบบทำได้อย่างถูกต้องหรือไม่
                \item การใช้งาน Community ทำได้อย่างถูกต้องหรือไม่
                \item การเลือกใช้ Map data ทำได้อย่างถูกต้องหรือไม่
                \item การจัดการข้อมูลใน Workspace ทำได้อย่างถูกต้องหรือไม่
                \item การสร้าง Configuration data ทำได้อย่างถูกต้องหรือไม่
                \item การกรอกข้อมูล Scenario data ทำได้อย่างถูกต้องหรือไม่
                \item การแสดงผลลัพธ์จากการจำลองสามารถแสดงผลได้อย่างถูกต้องหรือไม่
            \end{itemize}
        \end{mypara}

\end{mypara}

