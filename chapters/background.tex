\chapter{\ifenglish Background Knowledge and Theory\else ทฤษฎีที่เกี่ยวข้อง\fi}

\section{บทนำ}
  \sloppy\indent บทนี้เริ่มต้นด้วยการศึกษาทฤษฎี งานวิจัย และโครงงานที่เกี่ยวข้อง 
เพื่อเป็นแนวทางในการออกแบบและพัฒนาเว็บระบบสนับสนุนการตัดสินใจสำหรับการวางแผนระบบขนส่งสาธารณะ
โดยระบบดังกล่าวจะถูกออกแบบให้สามารถนำไปใช้กับ ระบบจำลอง (simulation) เพื่อให้ผลลัพธ์ใกล้เคียงกับของจริง
ซึ่งในบทนี้จะอธิบายเนื้อหาเกี่ยวกับแนวคิดและทฤษฎีที่เกี่ยวข้อง

\section{Platform Development}
\subsection{API Controller และ Logical Model}
\subsubsection{Service-Oriented Architecture (SOA) / Microservices}
\indent แนวทางการออกแบบสถาปัตยกรรมระบบซอฟต์แวร์โดยแยกฟังก์ชันการทำงานของระบบออกเป็น 
Services ที่เป็นอิสระจากกัน แต่สามารถสื่อสารและประสานงานกันได้ผ่านโปรโตคอลมาตรฐาน

\begin{itemize}
    \item \textbf{SOA} มุ่งเน้นการให้บริการ (Service) ในระดับองค์กร โดยมักจะใช้ ESB (Enterprise Service Bus) เป็นตัวกลาง
    \item \textbf{Microservices} เป็นวิวัฒนาการของ SOA ที่ลดความซับซ้อนของ ESB โดยให้แต่ละ Service เป็นอิสระเต็มที่ ใช้สื่อสารผ่าน API หรือ Event Driven
\end{itemize}

\begin{center}
\fbox{%
  \parbox{.8\textwidth}{%
งานวิจัยของ Newman (2015) เรื่อง "Building Microservices" และ Dragoni et al. (2017) เรื่อง "Microservices: yesterday, today, and tomorrow" อธิบายว่าการเปลี่ยนจาก SOA ไปเป็น Microservices ช่วยเพิ่มความยืดหยุ่น, scalability และ continuous delivery ได้ดีกว่าเดิม
  }%
}
\end{center}

\indent แนวคิดดังกล่าวจะถูกนํามาประยุกต์ใช้ในระบบของเราโดยในโครงสร้างระบบนี้ สามารถมองเป็นการใช้แนวคิด Microservices ดังนี้
\begin{enumerate}
    \item \textbf{API Controller (FastAPI)}  
    ทำหน้าที่เป็น Gateway Service หรือ API Gateway เป็นจุดรับและส่งข้อมูล (Request/Response) จากผู้ใช้หรือระบบภายนอก รวมถึงทำ Authentication, Load Balancing
    \item \textbf{Logical Model (Simulation/Calculation)}  
    ทำหน้าที่เป็น Processing Service รับข้อมูลจาก API Controller เพื่อทำ Simulation หรือการคำนวณเชิงตรรกะ สามารถรันเป็น Service แยก เช่นบน Container (Docker) หรือ Serverless Function เพื่อเพิ่ม scalability ได้
    \item \textbf{Frontend (Next.js / Vite)}  
    สื่อสารกับ Backend ผ่าน REST API สามารถ deploy หรืออัปเดต UI ได้โดยไม่กระทบ backend
\end{enumerate}

\subsubsection{RESTful API Design (Representational State Transfer)}
\indent รูปแบบการออกแบบ API สำหรับให้ระบบคอมพิวเตอร์สื่อสารกันผ่านเว็บ 
โดยแต่ละสิ่งที่ API จัดการจะถูกมองเป็นทรัพยากร (Resource) 
เช่น ข้อมูลผู้ใช้ ข้อมูล Simulation หรือเส้นทางเดินทาง โดยแต่ละทรัพยากรจะมี URL เฉพาะ 
และใช้ HTTP Methods (GET, POST, PUT, DELETE) เพื่อระบุการกระทำที่ต้องการกับทรัพยากรนั้น 
ระบบจะส่งข้อมูลกลับในรูปแบบที่อ่านง่าย เช่น JSON การใช้ RESTful API ช่วยให้ Frontend 
และ Backend แยกส่วนกันได้ชัดเจน เรียกใช้ Logic ที่ซับซ้อนผ่าน Request มาตรฐานได้ 
และทำให้การพัฒนาระบบง่ายต่อการเข้าใจ ใช้ซ้ำ และสเกลระบบได้ง่าย

\indent จากแนวคิดดังกล่าวจะถูกนํามาประยุกต์ใช้ในระบบของเราโดยจะมีประโยชน์ ดังนี้
\begin{enumerate}
    \item \textbf{ความง่ายในการเข้าถึงและเข้าใจ}  
    URL และ HTTP Methods มีความเป็นธรรมชาติ และสอดคล้องกับมาตรฐานสากล ทำให้ทีมพัฒนาต่างๆ เข้าใจได้ง่าย
    \item \textbf{การทำงานแบบ Stateless}  
    แต่ละ Request เป็นอิสระ ช่วยให้ระบบสามารถ scale ได้ง่ายขึ้น โดยเฉพาะในสภาพแวดล้อม Cloud หรือ Container
    \item \textbf{การบูรณาการกับ Frontend/Backend}  
    Frontend (Next.js / Vite) สามารถเรียกใช้ Logic ที่ซับซ้อน (เช่น Simulation Model) ผ่าน API แบบมาตรฐานโดยไม่ต้องรู้รายละเอียดเชิงลึกของ Backend
    \item \textbf{สนับสนุนมาตรฐาน HTTP และเครื่องมือทั่วไป}  
    เช่น curl, Postman ทำให้การทดสอบและบันทึกเอกสารทำได้สะดวก
    \item \textbf{ความยืดหยุ่นในการพัฒนาและปรับปรุง}  
    สามารถเพิ่ม Endpoint ใหม่หรือเปลี่ยนแปลง Backend ได้โดยไม่กระทบกับ Client หากยังคง Interface เดิม
\end{enumerate}

\subsection{Display Frontend}
\subsubsection{Web Mapping and GIS Fundamentals}
\indent Geographic Information Systems (GIS) เป็นระบบสารสนเทศที่ออกแบบมาเพื่อจัดเก็บ วิเคราะห์ 
และแสดงข้อมูลเชิงพื้นที่ (Spatial Data) GIS ใช้แนวคิดหลักหลายประการ เช่น การแทนข้อมูลเชิงพื้นที่ในรูปแบบ Vector 
และ Raster, การวิเคราะห์เชิงพื้นที่ (Spatial Analysis), การจัดการ Attribute Data 
และการสร้าง Visualization เพื่อสนับสนุนการตัดสินใจ
\begin{itemize}
    \item \textbf{Vector Data Model}: แสดงข้อมูลเชิงพื้นที่ด้วยจุด (Point), เส้น (Line), และพื้นที่ (Polygon)
    \begin{itemize}
        \item จุด (Point) ใช้แทนป้ายรถโดยสารหรือสถานีขนส่ง
        \item เส้น (Line) ใช้แทนเส้นทางการเดินรถหรือถนน
        \item พื้นที่ (Polygon) ใช้แทนเขตเมืองหรือบริเวณให้บริการ
    \end{itemize}
    \item \textbf{Raster Data Model}: แสดงพื้นที่เป็นกริดของค่าต่างๆ เช่น ความสูง, ความหนาแน่นประชากร
\end{itemize}

\begin{center}
\fbox{%
  \parbox{.8\textwidth}{%
Haklay et al., 2008, OpenStreetMap: User-Generated Street Maps แสดงว่าเว็บแผนที่สามารถใช้แสดงข้อมูลเชิงพื้นที่ที่ซับซ้อน
และสนับสนุนการตัดสินใจ เช่น การจัดสรรเส้นทางหรือการวางแผนระบบขนส่ง นอกจากนี้ 
Goodchild, 2007, Citizens as Sensors: The World of Volunteered Geography 
ชี้ให้เห็นว่า Web GIS สามารถผสมผสานข้อมูลจากผู้ใช้จริง(Crowdsourced Data) เพื่อสร้าง Heatmap 
หรือ Visual Analytics ของปัญหาการจราจรหรือการรอคอยผู้โดยสาร
  }%
}
\end{center}

\indent แนวคิดดังกล่าวจะถูกนํามาประยุกต์ใช้ในระบบของเราเพื่อช่วยการแสดงเครือข่ายขนส่งสาธารณะบนแผนที่เว็บ ดังนี้ 
\begin{enumerate}
    \item \textbf{Vector Data Representation}
    \begin{itemize}
        \item Nodes (จุด) แทนป้ายรถ, สถานีขนส่ง, จุดขึ้นลงผู้โดยสาร
        \item Links (เส้น) แทนเส้นทางของรถโดยสารหรือเส้นทางเดินรถ
        \item การแทนข้อมูลด้วย Vector ทำให้สามารถคำนวณเชิงพื้นที่ เช่น หาทางที่สั้นที่สุด (Shortest Path), การคำนวณระยะทาง, การวิเคราะห์เครือข่าย (Network Connectivity)
    \end{itemize}

    \item \textbf{Visualization}
    \begin{itemize}
        \item Map Display: Leaflet.js แสดงข้อมูลเชิงพื้นที่บนเว็บ ทำให้ผู้ใช้เห็นโครงสร้างเครือข่ายขนส่ง
        \item Heatmap / Thematic Map: แสดงข้อมูลเชิงปริมาณ เช่น พื้นที่ที่มีผู้โดยสารรอคอยสูง ความหนาแน่นผู้ใช้บริการ หรือเวลาเฉลี่ยการรอรถ
        \item Interactive Map: ผู้ใช้สามารถ Zoom, Pan, หรือ Click เพื่อตรวจสอบรายละเอียดของ Nodes/Links
    \end{itemize}
\end{enumerate}

\subsection{Database}
\subsubsection{Geospatial Database (PostGIS Extension for PostgreSQL)}
\indent Geospatial Database เป็นการขยายความสามารถของฐานข้อมูลเชิงสัมพันธ์ (RDBMS) 
ให้สามารถจัดเก็บ วิเคราะห์ และสืบค้นข้อมูลเชิงพื้นที่ (Spatial Data) ได้ 
ข้อมูลเชิงพื้นที่มักประกอบด้วยพิกัดทางภูมิศาสตร์ (Geographic Coordinates) 
เช่น จุด, เส้น, พื้นที่ ซึ่งแตกต่างจากข้อมูลทั่วไปที่เป็นตัวเลขหรือข้อความ
\begin{itemize}
    \item การจัดเก็บข้อมูลแบบ \textbf{Geometry} และ \textbf{Geography}
    \item การคำนวณเชิงพื้นที่ เช่น ระยะทาง, พื้นที่, การตัดกันของเส้นทาง (Intersections)
    \item การวิเคราะห์เครือข่าย เช่น หาทางที่สั้นที่สุด (Shortest Path), การเชื่อมต่อเครือข่าย
    \item การ Query Spatial เช่น การค้นหาวัตถุที่อยู่ในรัศมี X เมตร, การค้นหาจุดที่อยู่บนเส้นทางใด ๆ
\end{itemize}

\indent ได้นำแนวคิดมาประยุกต์ใช้กับระบบเครือข่ายขนส่งสาธารณะของเรา ดังนี้
\begin{enumerate}
    \item \textbf{การจัดเก็บข้อมูลเชิงพื้นที่ (Spatial Data Storage)}
    \begin{itemize}
        \item \textbf{Nodes}: ป้ายรถ, สถานีขนส่ง, จุดขึ้นลงผู้โดยสาร
        \item \textbf{Links}: เส้นทางของรถ, เส้นทางเดินรถ
        \item \textbf{Attribute Data}: ข้อมูลเสริม เช่น เวลาเปิด-ปิด, จำนวนผู้โดยสารเฉลี่ย, ประเภทเส้นทาง
    \end{itemize}

    \item \textbf{การคำนวณเชิงพื้นที่ (Spatial Computation)}
    \begin{itemize}
        \item คำนวณระยะทางระหว่าง Nodes หรือบน Links
        \item หาทางที่สั้นที่สุดระหว่างจุดต้นทางและปลายทาง
        \item วิเคราะห์ความเชื่อมโยงของเครือข่าย (Network Connectivity Analysis)
    \end{itemize}
    \item \textbf{การรวมกับ Web GIS และ Mapping Tools}
    \begin{itemize}
        \item ข้อมูลจาก PostGIS สามารถดึงไปใช้กับ Leaflet.js เพื่อสร้างแผนที่ Interactive และ Heatmap
        \item สนับสนุนการวิเคราะห์และ Visualization แบบ Real-time
    \end{itemize}
\end{enumerate}



% % define a command that produces some filler text, the lorem ipsum.
% \newcommand{\loremipsum}{
%   \textit{Lorem ipsum dolor sit amet, consectetur adipisicing elit, sed do
%   eiusmod tempor incididunt ut labore et dolore magna aliqua. Ut enim ad
%   minim veniam, quis nostrud exercitation ullamco laboris nisi ut
%   aliquip ex ea commodo consequat. Duis aute irure dolor in
%   reprehenderit in voluptate velit esse cillum dolore eu fugiat nulla
%   pariatur. Excepteur sint occaecat cupidatat non proident, sunt in
%   culpa qui officia deserunt mollit anim id est laborum.}\par}

% \begin{figure}
%   \centering

%   \fbox{
%      \parbox{.6\textwidth}{\loremipsum}
%   }

%   % To include an image in the figure, say myimage.pdf, you could use
%   % the following code. Look up the documentation for the package
%   % graphicx for more information.
%   % \includegraphics[width=\textwidth]{myimage}

%   \caption[Sample figure]{This figure is a sample containing \gls{lorem ipsum},
%   showing you how you can include figures and glossary in your report.
%   You can specify a shorter caption that will appear in the List of Figures.}
%   \label{fig:sample-figure}
% \end{figure}

% Using \verb.\label. and \verb.\ref. commands allows us to refer to
% figures easily. If we can refer to Figures
% \ref{fig:walrus} and \ref{fig:sample-figure} by name in the {\LaTeX}
% source code, then we will not need to update the code that refers to it
% even if the placement or ordering of the figures changes.

% \loremipsum\loremipsum

% This code demonstrates how to get a landscape table or figure. It
% uses the package lscape to turn everything but the page number into
% landscape orientation. Everything should be included within an
% \afterpage{ .... } to avoid causing a page break too early.
\afterpage{
  \begin{landscape}
  \begin{table}
    \caption{Sample landscape table}
    \label{tab:sample-table}

    \centering

    \begin{tabular}{c||c|c}
        Year & A & B \\
        \hline\hline
        1989 & 12 & 23 \\
        1990 & 4 & 9 \\
        1991 & 3 & 6 \\
    \end{tabular}
  \end{table}
  \end{landscape}
}

% \loremipsum\loremipsum\loremipsum

\section{\ifenglish%
\ifcpe CPE \else ISNE \fi knowledge used, applied, or integrated in this project
\else%
ความรู้ตามหลักสูตรซึ่งถูกนำมาใช้หรือบูรณาการในโครงงาน
\fi
}

อธิบายถึงความรู้ และแนวทางการนำความรู้ต่างๆ ที่ได้เรียนตามหลักสูตร ซึ่งถูกนำมาใช้ในโครงงาน

\section{\ifenglish%
Extracurricular knowledge used, applied, or integrated in this project
\else%
ความรู้นอกหลักสูตรซึ่งถูกนำมาใช้หรือบูรณาการในโครงงาน
\fi
}

อธิบายถึงความรู้ต่างๆ ที่เรียนรู้ด้วยตนเอง และแนวทางการนำความรู้เหล่านั้นมาใช้ในโครงงาน
