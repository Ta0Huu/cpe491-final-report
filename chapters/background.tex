\chapter{\ifenglish Background Knowledge and Theory\else ทฤษฎีที่เกี่ยวข้อง\fi}

\section{บทนำ}
  \sloppy\indent ในบทนี้จะนำเสนอการศึกษาทฤษฎีและงานวิจัยที่เกี่ยวข้อง เพื่อเป็นแนวทางในการออกแบบและพัฒนาเว็บแอปพลิเคชันสำหรับระบบสนับสนุนการตัดสินใจในการวางแผนระบบขนส่งสาธารณะ 
  ระบบดังกล่าวถูกออกแบบให้ใช้การจำลอง (simulation) เพื่อสร้างสถานการณ์บนแผนต่างๆ 
  และให้ผลลัพธ์ใกล้เคียงกับความเป็นจริง ในบทนี้จะอธิบายแนวคิดและทฤษฎีที่เกี่ยวข้องกับการพัฒนาระบบดังกล่าว

\section{Platform Development}

\subsection{Frontend Display}
\subsubsection{Vite}
\indent Vite เป็น Frontend Build Tool และ Development Server ที่ออกแบบเพื่อปรับปรุงประสบการณ์การพัฒนาเว็บแอปพลิเคชันสมัยใหม่ มีจุดเด่นเรื่อง \textbf{ความเร็วสูง} และ \textbf{HMR (Hot Module Replacement) แบบเรียลไทม์} ทำให้การพัฒนา Frontend ราบรื่นและสนุกยิ่งขึ้น  

\begin{itemize}
    \item \textbf{Native ES Modules}: ใช้ฟีเจอร์ ES Modules ของเบราว์เซอร์ ทำให้ไม่ต้อง Bundle ทั้งโปรเจกต์ในระหว่าง Development
    \item \textbf{Instant Server Start}: เริ่ม Development Server ได้ทันทีโดยไม่ต้องรอ Build
    \item \textbf{Hot Module Replacement (HMR)}: เปลี่ยนแปลงโค้ดแล้วเห็นผลทันทีโดยไม่โหลดหน้าใหม่
    \item \textbf{Optimized Production Build}: ใช้ Rollup สำหรับสร้างไฟล์ Production ที่มีประสิทธิภาพและขนาดเล็ก
\end{itemize}

ในระบบสนับสนุนการตัดสินใจด้านขนส่งสาธารณะ Vite ถูกใช้เป็น Frontend Build Tool สำหรับ Next.js เพื่อพัฒนา Interface ที่โต้ตอบกับ Backend (Go Fiber) 
และแสดงผลข้อมูลเชิงพื้นที่/Simulation Model ดังนี้

\begin{enumerate}
    \item \textbf{Development Server และ HMR}
    \begin{itemize}
        \item พัฒนา UI/UX แบบ Interactive เช่น Map (Leaflet.js), Graph, และ Dashboard
        \item เปลี่ยนแปลงโค้ดแล้วเห็นผลทันที ทำให้ปรับ Layout หรือ Data Visualization ได้รวดเร็ว
    \end{itemize}
    
    \item \textbf{Integration กับ React / Next.js}
    \begin{itemize}
        \item ใช้ Vite เป็นเครื่องมือ Build สำหรับ Next.js/React Component
        \item จัดการ Asset, CSS Module, และ ES Modules ได้ง่าย  
        \item รองรับการทำ Code Splitting เพื่อลดขนาด Bundle และเพิ่ม Performance
    \end{itemize}
    
    \item \textbf{Data Fetching และ API Consumption}
    \begin{itemize}
        \item Frontend ติดต่อกับ Backend (Go Fiber) ผ่าน RESTful API  
        \item ดึงผลลัพธ์จาก Simulation Model หรือ PostGIS Spatial Query มาสร้าง Heatmap, Graph, หรือ Table แบบ Real-time
    \end{itemize}
    
    \item \textbf{Optimized Production}
    \begin{itemize}
        \item Build ไฟล์ Production ขนาดเล็ก, Tree-Shaking, และ Minification  
        \item ทำให้โหลดหน้าเว็บเร็วขึ้นแม้ข้อมูลเชิงพื้นที่หรือ Graph มีขนาดใหญ่
    \end{itemize}
\end{enumerate}

\subsubsection{TypeScript}
TypeScript เป็นภาษาโปรแกรมที่พัฒนาโดย Microsoft ซึ่งเป็น \textbf{superset ของ JavaScript} เพิ่มความสามารถด้าน \textbf{Static Typing} และ \textbf{Compile-Time Error Checking} ทำให้โค้ดมีความปลอดภัยและลดข้อผิดพลาดในระหว่างการพัฒนา  
\begin{itemize}
    \item \textbf{Static Typing}: ประกาศประเภทของตัวแปร, ฟังก์ชัน, และ Object ช่วยตรวจสอบข้อผิดพลาดก่อนรัน
    \item \textbf{Enhanced IDE Support}: รองรับ IntelliSense, Auto-completion, และ Refactoring
    \item \textbf{Compatibility}: สามารถรันโค้ด JavaScript ได้ทั้งหมด และ Compile เป็น JavaScript สำหรับ Browser หรือ Node.js
    \item \textbf{Object-Oriented Features}: รองรับ Class, Interface, Generics และ Module System
\end{itemize}

TypeScript ถูกใช้เป็นภาษาหลักในการพัฒนา Frontend ของระบบ ด้วย Next.js และ Vite Framework เพื่อสร้าง Interface ที่โต้ตอบกับ Backend (Go Fiber) และ Visualize ผลลัพธ์ของ Simulation Model / GIS Data
\begin{enumerate}
    \item \textbf{Static Typing for API Data}
    \begin{itemize}
        \item กำหนด Interface สำหรับ Response จาก Backend เช่น Simulation Result, Spatial Data
        \item ลดข้อผิดพลาดจากการเรียกใช้ JSON หรือ Object Property
    \end{itemize}
    
    \item \textbf{Component-Based Development}
    \begin{itemize}
        \item สร้าง React Component แบบ Typed เพื่อ Map, Graph, Table
        \item ใช้ Props และ State แบบ Type-Safe
    \end{itemize}
    
    \item \textbf{Improved Maintainability}
    \begin{itemize}
        \item ทำให้ทีมสามารถ Refactor โค้ดได้ง่ายและปลอดภัย
        \item รองรับการพัฒนา Feature ใหม่หรือปรับ UI/UX โดยไม่ทำลายระบบเดิม
    \end{itemize}
    
    \item \textbf{Integration with Vite / Next.js}
    \begin{itemize}
        \item Vite รองรับการ Compile TypeScript ได้โดยตรง
        \item TypeScript ช่วยให้การเชื่อมต่อ API, State Management, และ Component Interaction มีความปลอดภัยและมีประสิทธิภาพ
    \end{itemize}
\end{enumerate}

\subsection{Logical Model}
\subsubsection{Salabim}
Salabim เป็น Python-based Discrete-Event Simulation (DES) framework ที่ออกแบบมาเพื่อสร้างและรันระบบจำลองเชิงเหตุการณ์ (Event-Driven Simulation) ได้อย่างยืดหยุ่นและชัดเจน 
แนวคิดหลักของ Salabim ได้แก่:
\begin{itemize}
    \item \textbf{Discrete-Event Simulation}: ระบบเปลี่ยนแปลงสถานะเมื่อเกิดเหตุการณ์ (Event) เช่น การมาถึงของผู้โดยสาร การเคลื่อนที่ของรถ
    \item \textbf{Process-Oriented Modeling}: ใช้ Process class สำหรับกำหนดพฤติกรรมของ Entities (เช่น รถโดยสาร)
    \item \textbf{Resources}: สามารถกำหนด Resource, Queue, Priority Queue และเงื่อนไขการเข้าใช้
    \item \textbf{Stochastic Modeling}: สนับสนุนการสุ่มเหตุการณ์และเวลาในการรอ เช่น Arrival Time, Service Time
\end{itemize}

\subsubsection{แนวคิดดังกล่าวจะถูกนํามาประยุกต์ใช้ในระบบของเรา โดยใช้ใน Logical Model ดังนี้}
\begin{enumerate}
    \item \textbf{การจำลองเหตุการณ์ของผู้โดยสารและรถ}
    \begin{itemize}
        \item Entities: รถโดยสาร, ผู้โดยสาร, ป้ายรถ
        \item Events: Arrival, Departure, Boarding, Alighting
        \item Process: กำหนดพฤติกรรม เช่น รถวิ่งตามเส้นทาง, ผู้โดยสารขึ้น/ลง
    \end{itemize}
    
    \item \textbf{การจัดการทรัพยากร (Resource Management)}
    \begin{itemize}
        \item กำหนดจำนวนรถแต่ละเส้นทาง
        \item วิเคราะห์ปัญหาเวลารอของผู้โดยสาร
    \end{itemize}
    
    \item \textbf{การวิเคราะห์เชิงสถิติและผลลัพธ์}
    \begin{itemize}
        \item Collect Metrics: เวลาเฉลี่ยรอรถ, จำนวนผู้โดยสาร, ความหนาแน่น
        \item Visualization: ส่งผลลัพธ์ไปยัง Heatmap หรือ Graph ของ Frontend
        \item Scenario Analysis: ปรับจำนวนรถ, ตารางเวลา หรือเส้นทางเพื่อเปรียบเทียบประสิทธิภาพ
    \end{itemize}
\end{enumerate}

\subsection{API Controller}
% \subsubsection{Service-Oriented Architecture (SOA)}
% \indent แนวทางการออกแบบสถาปัตยกรรมระบบซอฟต์แวร์โดยแยกฟังก์ชันการทำงานของระบบออกเป็น 
% Services ที่เป็นอิสระจากกัน แต่สามารถสื่อสารและประสานงานกันได้ผ่านโปรโตคอลมาตรฐาน

% \begin{itemize}
%     \item \textbf{SOA} มุ่งเน้นการให้บริการ (Service) ในระดับองค์กร โดยมักจะใช้ ESB (Enterprise Service Bus) เป็นตัวกลาง
% \end{itemize}

% \indent แนวคิดดังกล่าวจะถูกนํามาประยุกต์ใช้ในระบบของเราโดยในโครงสร้างระบบ ดังนี้
% \begin{enumerate}
%     \item \textbf{API Controller (FastAPI)}  
%     ทำหน้าที่เป็น Gateway Service หรือ API Gateway เป็นจุดรับและส่งข้อมูล (Request/Response) จากผู้ใช้หรือระบบภายนอก รวมถึงทำ Authentication, Load Balancing
%     \item \textbf{Logical Model (Simulation/Calculation)}  
%     ทำหน้าที่เป็น Processing Service รับข้อมูลจาก API Controller เพื่อทำ Simulation หรือการคำนวณเชิงตรรกะ สามารถรันเป็น Service แยก เช่นบน Container (Docker) หรือ Serverless Function เพื่อเพิ่ม scalability ได้
%     \item \textbf{Frontend (Next.js / Vite)}  
%     สื่อสารกับ Backend ผ่าน REST API สามารถ deploy หรืออัปเดต UI ได้โดยไม่กระทบ backend
% \end{enumerate}

\subsubsection{RESTful API Design (Representational State Transfer)}
\indent รูปแบบการออกแบบ API สำหรับให้ระบบคอมพิวเตอร์สื่อสารกันผ่านเว็บ 
โดยแต่ละสิ่งที่ API จัดการจะถูกมองเป็นทรัพยากร (Resource) 
เช่น ข้อมูลผู้ใช้ ข้อมูล Simulation หรือเส้นทางเดินทาง โดยแต่ละทรัพยากรจะมี URL เฉพาะ 
และใช้ HTTP Methods (GET, POST, PUT, DELETE) เพื่อระบุการกระทำที่ต้องการกับทรัพยากรนั้น 
ระบบจะส่งข้อมูลกลับในรูปแบบที่อ่านง่าย เช่น JSON การใช้ RESTful API ช่วยให้ Frontend 
และ Backend แยกส่วนกันได้ชัดเจน เรียกใช้ Logic ที่ซับซ้อนผ่าน Request มาตรฐานได้ 
และทำให้การพัฒนาระบบง่ายต่อการเข้าใจ ใช้ซ้ำ และสเกลระบบได้ง่าย

\subsubsection{จากแนวคิดดังกล่าวจะถูกนํามาประยุกต์ใช้ในระบบของเราโดยจะมีประโยชน์ ดังนี้}
\begin{enumerate}
    \item \textbf{ความง่ายในการเข้าถึงและเข้าใจ}  
    URL และ HTTP Methods มีความเป็นธรรมชาติ และสอดคล้องกับมาตรฐานสากล ทำให้ทีมพัฒนาต่างๆ เข้าใจได้ง่าย
    \item \textbf{การทำงานแบบ Stateless}  
    แต่ละ Request เป็นอิสระ ช่วยให้ระบบสามารถ scale ได้ง่ายขึ้น โดยเฉพาะในสภาพแวดล้อม Cloud หรือ Container
    \item \textbf{การบูรณาการกับ Frontend/Backend}  
    Frontend (Next.js / Vite) สามารถเรียกใช้ Logic ที่ซับซ้อน (เช่น Simulation Model) ผ่าน API แบบมาตรฐานโดยไม่ต้องรู้รายละเอียดเชิงลึกของ Backend
    \item \textbf{สนับสนุนมาตรฐาน HTTP และเครื่องมือทั่วไป}  
    เช่น curl, Postman ทำให้การทดสอบและบันทึกเอกสารทำได้สะดวก
    \item \textbf{ความยืดหยุ่นในการพัฒนาและปรับปรุง}  
    สามารถเพิ่ม Endpoint ใหม่หรือเปลี่ยนแปลง Backend ได้โดยไม่กระทบกับ Client หากยังคง Interface เดิม
\end{enumerate}

\subsubsection{Go Fiber Framework}
\indent Go Fiber เป็น Web Framework สำหรับภาษา Golang ที่ได้รับแรงบันดาลใจจาก Express.js ของ Node.js มีจุดเด่นคือ ความเร็วสูง, น้ำหนักเบา และใช้งานง่าย เหมาะกับการสร้าง RESTful API, Microservices และ Web Application โดย Fiber ใช้ fasthttp เป็น HTTP engine ทำให้สามารถประมวลผล Request/Response ได้เร็วมากเมื่อเทียบกับ Framework อื่นๆ  

\subsubsection{คุณสมบัติเด่นของ Go Fiber ได้แก่:}
\begin{itemize}
    \item \textbf{Performance}: รองรับ Request จำนวนมากด้วย Latency ต่ำ  
    \item \textbf{Minimalism}: API เรียบง่าย เรียนรู้ง่ายและใกล้เคียงกับ Express.js  
    \item \textbf{Routing}: รองรับ Dynamic Routing, Group Routing, Middleware  
    \item \textbf{Middleware}: สามารถเพิ่ม Logging, CORS, Authentication, Compression, และอื่น ๆ  
    \item \textbf{WebSocket Support}: รองรับการสื่อสารแบบ Real-time  
\end{itemize}

\subsubsection{ประโยชน์ของ Go Fiber ในระบบของเรา}
\begin{enumerate}
    \item ความเร็วสูง ลดเวลา Response ของ API สำหรับ Simulation Model
    \item API เรียบง่ายและ Modular สามารถจัดการ Routing และ Middleware แยกตาม Module ได้
    \item รองรับการ Scale ระบบในอนาคต เช่น เพิ่ม Microservice หรือ Containerize ด้วย Docker
    \item ทำงานร่วมกับ Database และ Web GIS ได้อย่างราบรื่น
    \item สนับสนุนการพัฒนาแบบ Concurrent และ Real-time เช่น WebSocket สำหรับ Live Updates
\end{enumerate}



%\subsection{Display Frontend}
% \subsubsection{Web Mapping and GIS Fundamentals}
% \indent Geographic Information Systems (GIS) เป็นระบบสารสนเทศที่ออกแบบมาเพื่อจัดเก็บ วิเคราะห์ 
% และแสดงข้อมูลเชิงพื้นที่ (Spatial Data) GIS ใช้แนวคิดหลักหลายประการ เช่น การแทนข้อมูลเชิงพื้นที่ในรูปแบบ Vector 
% และ Raster, การวิเคราะห์เชิงพื้นที่ (Spatial Analysis), การจัดการ Attribute Data 
% และการสร้าง Visualization เพื่อสนับสนุนการตัดสินใจ
% \begin{itemize}
%     \item \textbf{Vector Data Model}: แสดงข้อมูลเชิงพื้นที่ด้วยจุด (Point), เส้น (Line), และพื้นที่ (Polygon)
%     \begin{itemize}
%         \item จุด (Point) ใช้แทนป้ายรถโดยสารหรือสถานีขนส่ง
%         \item เส้น (Line) ใช้แทนเส้นทางการเดินรถหรือถนน
%         \item พื้นที่ (Polygon) ใช้แทนเขตเมืองหรือบริเวณให้บริการ
%     \end{itemize}
%     \item \textbf{Raster Data Model}: แสดงพื้นที่เป็นกริดของค่าต่างๆ เช่น ความสูง, ความหนาแน่นประชากร
% \end{itemize}

% \begin{center}
% \fbox{%
%   \parbox{.8\textwidth}{%
% Haklay et al., 2008, OpenStreetMap: User-Generated Street Maps แสดงว่าเว็บแผนที่สามารถใช้แสดงข้อมูลเชิงพื้นที่ที่ซับซ้อน
% และสนับสนุนการตัดสินใจ เช่น การจัดสรรเส้นทางหรือการวางแผนระบบขนส่ง นอกจากนี้ 
% Goodchild, 2007, Citizens as Sensors: The World of Volunteered Geography 
% ชี้ให้เห็นว่า Web GIS สามารถผสมผสานข้อมูลจากผู้ใช้จริง(Crowdsourced Data) เพื่อสร้าง Heatmap 
% หรือ Visual Analytics ของปัญหาการจราจรหรือการรอคอยผู้โดยสาร
%   }%
% }
% \end{center}

% \indent แนวคิดดังกล่าวจะถูกนํามาประยุกต์ใช้ในระบบของเราเพื่อช่วยการแสดงเครือข่ายขนส่งสาธารณะบนแผนที่เว็บ ดังนี้ 
% \begin{enumerate}
%     \item \textbf{Vector Data Representation}
%     \begin{itemize}
%         \item Nodes (จุด) แทนป้ายรถ, สถานีขนส่ง, จุดขึ้นลงผู้โดยสาร
%         \item Links (เส้น) แทนเส้นทางของรถโดยสารหรือเส้นทางเดินรถ
%         \item การแทนข้อมูลด้วย Vector ทำให้สามารถคำนวณเชิงพื้นที่ เช่น หาทางที่สั้นที่สุด (Shortest Path), การคำนวณระยะทาง, การวิเคราะห์เครือข่าย (Network Connectivity)
%     \end{itemize}

%     \item \textbf{Visualization}
%     \begin{itemize}
%         \item Map Display: Leaflet.js แสดงข้อมูลเชิงพื้นที่บนเว็บ ทำให้ผู้ใช้เห็นโครงสร้างเครือข่ายขนส่ง
%         \item Heatmap / Thematic Map: แสดงข้อมูลเชิงปริมาณ เช่น พื้นที่ที่มีผู้โดยสารรอคอยสูง ความหนาแน่นผู้ใช้บริการ หรือเวลาเฉลี่ยการรอรถ
%         \item Interactive Map: ผู้ใช้สามารถ Zoom, Pan, หรือ Click เพื่อตรวจสอบรายละเอียดของ Nodes/Links
%     \end{itemize}
% \end{enumerate}

%\subsection{Database}
% \subsubsection{Geospatial Database (PostGIS Extension for PostgreSQL)}
% \indent Geospatial Database เป็นการขยายความสามารถของฐานข้อมูลเชิงสัมพันธ์ (Relational Database Management System) 
% ให้สามารถจัดเก็บ วิเคราะห์ และสืบค้นข้อมูลเชิงพื้นที่ (Spatial Data) ได้ 
% ข้อมูลเชิงพื้นที่มักประกอบด้วยพิกัดทางภูมิศาสตร์ (Geographic Coordinates) 
% เช่น จุด, เส้น, พื้นที่ ซึ่งแตกต่างจากข้อมูลทั่วไปที่เป็นตัวเลขหรือข้อความ
% \begin{itemize}
%     \item การจัดเก็บข้อมูลแบบ \textbf{Geometry} และ \textbf{Geography}
%     \item การคำนวณเชิงพื้นที่ เช่น ระยะทาง, พื้นที่, การตัดกันของเส้นทาง (Intersections)
%     \item การวิเคราะห์เครือข่าย เช่น หาทางที่สั้นที่สุด (Shortest Path), การเชื่อมต่อเครือข่าย
%     \item การ Query Spatial เช่น การค้นหาวัตถุที่อยู่ในรัศมี X เมตร, การค้นหาจุดที่อยู่บนเส้นทางใด ๆ
% \end{itemize}

% \indent ได้นำแนวคิดมาประยุกต์ใช้กับระบบเครือข่ายขนส่งสาธารณะของเรา ดังนี้
% \begin{enumerate}
%     \item \textbf{การจัดเก็บข้อมูลเชิงพื้นที่ (Spatial Data Storage)}
%     \begin{itemize}
%         \item \textbf{Nodes}: ป้ายรถ, สถานีขนส่ง, จุดขึ้นลงผู้โดยสาร
%         \item \textbf{Links}: เส้นทางของรถ, เส้นทางเดินรถ
%         \item \textbf{Attribute Data}: ข้อมูลเสริม เช่น เวลาเปิด-ปิด, จำนวนผู้โดยสารเฉลี่ย, ประเภทเส้นทาง
%     \end{itemize}

%     \item \textbf{การคำนวณเชิงพื้นที่ (Spatial Computation)}
%     \begin{itemize}
%         \item คำนวณระยะทางระหว่าง Nodes หรือบน Links
%         \item หาทางที่สั้นที่สุดระหว่างจุดต้นทางและปลายทาง
%         \item วิเคราะห์ความเชื่อมโยงของเครือข่าย (Network Connectivity Analysis)
%     \end{itemize}
%     \item \textbf{การรวมกับ Web GIS และ Mapping Tools}
%     \begin{itemize}
%         \item ข้อมูลจาก PostGIS สามารถดึงไปใช้กับ Leaflet.js เพื่อสร้างแผนที่ Interactive และ Heatmap
%         \item สนับสนุนการวิเคราะห์และ Visualization แบบ Real-time
%     \end{itemize}
% \end{enumerate}


\afterpage{
  \begin{landscape}
  \begin{table}
    \caption{Sample landscape table}
    \label{tab:sample-table}

    \centering

    \begin{tabular}{c||c|c}
        Year & A & B \\
        \hline\hline
        1989 & 12 & 23 \\
        1990 & 4 & 9 \\
        1991 & 3 & 6 \\
    \end{tabular}
  \end{table}
  \end{landscape}
}

\section{\ifenglish%
\ifcpe CPE \else ISNE \fi knowledge used, applied, or integrated in this project
\else%
ความรู้ตามหลักสูตรซึ่งถูกนำมาใช้หรือบูรณาการในโครงงาน
\fi
}

อธิบายถึงความรู้ และแนวทางการนำความรู้ต่างๆ ที่ได้เรียนตามหลักสูตร ซึ่งถูกนำมาใช้ในโครงงาน

\section{\ifenglish%
Extracurricular knowledge used, applied, or integrated in this project
\else%
ความรู้นอกหลักสูตรซึ่งถูกนำมาใช้หรือบูรณาการในโครงงาน
\fi
}

อธิบายถึงความรู้ต่างๆ ที่เรียนรู้ด้วยตนเอง และแนวทางการนำความรู้เหล่านั้นมาใช้ในโครงงาน
