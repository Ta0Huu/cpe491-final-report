\chapter{\ifenglish Background Knowledge and Theory\else ทฤษฎีที่เกี่ยวข้อง\fi}

\section{บทนำ}
  \sloppy บทนี้เริ่มต้นด้วยการศึกษาทฤษฎี งานวิจัย และโครงงานที่เกี่ยวข้อง 
เพื่อเป็นแนวทางในการออกแบบและพัฒนาเว็บระบบสนับสนุนการตัดสินใจสำหรับการวางแผนระบบขนส่งสาธารณะ
โดยระบบดังกล่าวจะถูกออกแบบให้สามารถนำไปใช้กับ ระบบจำลอง (simulation) เพื่อให้ผลลัพธ์ใกล้เคียงกับของจริง
ซึ่งในบทนี้จะอธิบายเนื้อหาเกี่ยวกับแนวคิดและทฤษฎีที่เกี่ยวข้อง

\section{Platform Development}
\subsection{API Controller และ Logical Model}
\subsubsection{Service-Oriented Architecture (SOA)}
Section 3 text. The dielectric constant\index{dielectric constant}
at the air-metal interface determines
the resonance shift\index{resonance shift} as absorption or capture occurs
is shown in Equation~\eqref{eq:dielectric}:

\section{ความรู้ตามหลักสูตรซึ่งถูกนํามาใช้หรือบูรณาการในโครงงาน}

% define a command that produces some filler text, the lorem ipsum.
\newcommand{\loremipsum}{
  \textit{Lorem ipsum dolor sit amet, consectetur adipisicing elit, sed do
  eiusmod tempor incididunt ut labore et dolore magna aliqua. Ut enim ad
  minim veniam, quis nostrud exercitation ullamco laboris nisi ut
  aliquip ex ea commodo consequat. Duis aute irure dolor in
  reprehenderit in voluptate velit esse cillum dolore eu fugiat nulla
  pariatur. Excepteur sint occaecat cupidatat non proident, sunt in
  culpa qui officia deserunt mollit anim id est laborum.}\par}

\begin{figure}
  \centering

  \fbox{
     \parbox{.6\textwidth}{\loremipsum}
  }

  % To include an image in the figure, say myimage.pdf, you could use
  % the following code. Look up the documentation for the package
  % graphicx for more information.
  % \includegraphics[width=\textwidth]{myimage}

  \caption[Sample figure]{This figure is a sample containing \gls{lorem ipsum},
  showing you how you can include figures and glossary in your report.
  You can specify a shorter caption that will appear in the List of Figures.}
  \label{fig:sample-figure}
\end{figure}

Using \verb.\label. and \verb.\ref. commands allows us to refer to
figures easily. If we can refer to Figures
\ref{fig:walrus} and \ref{fig:sample-figure} by name in the {\LaTeX}
source code, then we will not need to update the code that refers to it
even if the placement or ordering of the figures changes.

\loremipsum\loremipsum

% This code demonstrates how to get a landscape table or figure. It
% uses the package lscape to turn everything but the page number into
% landscape orientation. Everything should be included within an
% \afterpage{ .... } to avoid causing a page break too early.
\afterpage{
  \begin{landscape}
  \begin{table}
    \caption{Sample landscape table}
    \label{tab:sample-table}

    \centering

    \begin{tabular}{c||c|c}
        Year & A & B \\
        \hline\hline
        1989 & 12 & 23 \\
        1990 & 4 & 9 \\
        1991 & 3 & 6 \\
    \end{tabular}
  \end{table}
  \end{landscape}
}

\loremipsum\loremipsum\loremipsum

\section{Overfull hbox}

When the \verb.semifinal. option is passed to the \verb.cpecmu. document class,
any line that is longer than the line width, i.e., an overfull hbox, will be
highlighted with a black solid rule:
\begin{center}
\begin{minipage}{2em}
juxtaposition
\end{minipage}
\end{center}

\section{\ifenglish%
\ifcpe CPE \else ISNE \fi knowledge used, applied, or integrated in this project
\else%
ความรู้ตามหลักสูตรซึ่งถูกนำมาใช้หรือบูรณาการในโครงงาน
\fi
}

อธิบายถึงความรู้ และแนวทางการนำความรู้ต่างๆ ที่ได้เรียนตามหลักสูตร ซึ่งถูกนำมาใช้ในโครงงาน

\section{\ifenglish%
Extracurricular knowledge used, applied, or integrated in this project
\else%
ความรู้นอกหลักสูตรซึ่งถูกนำมาใช้หรือบูรณาการในโครงงาน
\fi
}

อธิบายถึงความรู้ต่างๆ ที่เรียนรู้ด้วยตนเอง และแนวทางการนำความรู้เหล่านั้นมาใช้ในโครงงาน
