\chapter{\ifproject%
\ifenglish Project Structure and Methodology\else โครงสร้างและขั้นตอนการทำงาน\fi
\else%
\ifenglish Project Structure\else โครงสร้างของโครงงาน\fi
\fi
}

\makeatletter

% \renewcommand\section{\@startsection {section}{1}{\z@}%
%                                    {13.5ex \@plus -1ex \@minus -.2ex}%
%                                    {2.3ex \@plus.2ex}%
%                                    {\normalfont\large\bfseries}}

\makeatother
%\vspace{2ex}
% \titleformat{\section}{\normalfont\bfseries}{\thesection}{1em}{}
% \titlespacing*{\section}{0pt}{10ex}{0pt}

\section{โครงสร้างโดยรวมของโครงงาน (Project Overview) }
\begin{mypara}
    \indent โครงงานนี้เป็นระบบสนับสนุนการตัดสินใจสำหรับการวางแผนระบบขนส่งสาธารณะ 
    โดยจะจัดทำเป็นเว็บแอพลิเคชั่นสำหรับการจำลอง และมีการสรุปผลลัพธ์ที่ได้จากการจำลองในรูปแบบทางสถิติ
    ซึ่งในการทำงานของระบบนี้จะมี 4 องค์ประกอบหลักดังนี้ที่ประกอบกันเป็นระบบเดียวกัน ดังภาพ \ref{fig:overview}  
\end{mypara}


\begin{figure}[H]
    \centering
    \includegraphics[width=\textwidth,height=0.9\textheight,keepaspectratio]{overview.png}
    \caption{แผนภาพโดยรวมของโครงงาน}
    \label{fig:overview}
\end{figure}

\subsection{ข้อมูลสถานการณ์ (Scenario data)}
\begin{mypara}
  \indent เป็นข้อมูลที่ผู้ใช้สามารถปรับเปลี่ยนพารามิเตอร์ได้บ่อย ๆ เพื่อจำลองสถานการณ์ต่าง ๆ 
  สำหรับการเปรียบเทียบและวางแผน ประกอบด้วย เส้นทางการให้บริการของรถในแต่ละสายบริการ 
  ตารางเวลาการออกรถ ช่วงเวลาที่ต้องการดูผลการจําลอง ข้อมูลของรถซึ่งมี 
  ความจุผู้โดยสารระยะทางที่สามารถวิ่งได้สูงสุด ความเร็วในการวิ่ง
\end{mypara}
\subsection{ข้อมูลการกำหนดค่า (Configuration data)}
\begin{mypara}
  \indent เป็นส่วนของข้อมูลที่ใช้เป็นพื้นฐานในการจำลองซึ่งจะไม่ค่อยมีการเปลี่ยนแปลงบ่อย 
  ได้แก่ ข้อมูลช่วงระยะห่างของเวลาที่ผู้โดยสารแต่ละคนมาถึงที่สถานี ข้อมูลของจำนวนผู้โดยสารที่ลงจากรถในแต่ละสถานี 
  และข้อมูลที่เกี่ยวข้องกับสถานีซึ่งจะเก็บในรูปแบบของ network model ซึ่งจะมี 
  node เป็นรายชื่อสถานี และ มี edge เป็นระยะทางละหว่างแต่ละสถานี
  \end{mypara}
\subsection{ส่วนการจำลอง (Simulation Component)}
\begin{mypara}
  \indent เป็นส่วนที่ทำหน้าที่ในการจำลองระบบขนส่งสาธารณะตามข้อมูลที่ได้รับจากส่วน input 
  และ configuration data ซึ่งจะใช้เทคนิคการจำลองแบบเหตุการณ์ไม่ต่อเนื่อง (discrete-event simulation) 
  ในการจำลองระบบขนส่งสาธารณะ 
\end{mypara}
\subsection{ผลลัพธ์ (Output)}
\begin{mypara}
  \indent เป็นส่วนที่แสดงผลลัพธ์ที่ได้จากการจำลอง โดยจะมีข้อมูล การรอเฉลี่ยของผู้โดยสาร อัตราการใช้งานของรถ 
  และอัตราการเกิดเหตุการณ์ที่รถบัสมีการใช้งานเกินเวลาหรือระยะทางที่กำหนด
  ซึ่งจะมีข้อมูลแสดงทั้งข้อมูลรวมทุกสถานี และข้อมูลรายละเอียดของแต่ละสถานีแยกกัน รวมถึงจะมี dashboard 
  สำหรับแสดงผลลัพธ์ในรูปแบบกราฟเพื่อให้ผู้ใช้สามารถวิเคราะห์ข้อมูลได้ง่ายขึ้น 
\end{mypara}
\section{ขั้นตอนการทำงานของระบบ (System Workflow)}
\begin{mypara}
    \indent โครงงานนี้ทำงานโดยนำข้อมูลสถานการณ์จากผู้ใช้ (Scenario data) มาจำลองระบบขนส่งสาธารณะ 
    โดยอ้างอิงจากข้อมูลการกำหนดค่า (configuration data) ซึ่งประกอบด้วยข้อมูลหลายประเภท 
    โดยบางส่วนจะถูกนำไปสร้างเป็น ฟังก์ชันการแจกแจง (Distribution Function) เพื่อใช้ในการสุ่มพฤติกรรมของผู้โดยสาร 
    ขณะที่ข้อมูลอีกส่วนจะถูกนำมาใช้โดยตรง ใน ส่วนการจำลอง ซึ่งจะประมวลผลด้วย
    เทคนิคการจำลองแบบเหตุการณ์ไม่ต่อเนื่อง (discrete-event simulation) 
    เพื่อสร้างผลลัพธ์เชิงสถิติที่สะท้อนประสิทธิภาพของระบบขนส่งสาธารณะมาแสดงผลในส่วน output
\end{mypara}

\section{ส่วนติดต่อผู้ใช้ (User Interface)}
\begin{mypara}
    \indent โครงงานนี้มีการออกแบบเพื่อให้ผู้ใช้สามารถป้อนข้อมูลที่จำเป็นสำหรับ
    การจำลองระบบขนส่งสาธารณะได้อย่างสะดวกและไม่ซับซ้อน 
    โดยข้อมูลที่มีการเปลี่ยนแปลงบ่อย จะถูกจัดให้อยู่ในส่วนของ input 
    ซึ่งสามารถเข้าถึงและแก้ไขได้ง่ายโดยไม่ต้องผ่านขั้นตอนที่ซับซ้อน 
    ส่วนข้อมูลพื้นฐานของระบบขนส่งสาธารณะ จะถูกจัดเก็บแยกเป็น configuration data 
    เพื่อให้ง่ายต่อการนำมาใช้ซ้ำ และมี workspace สำหรับการจัดการข้อมูลที่ช่วยให้ผู้ใช้สามารถ 
    clone configuration data หรือ scenario data จากผู้อื่นมาใช้งานและปรับแก้ได้สะดวก

  \indent ในส่วนของ output ได้ออกแบบให้สามารถแสดงผลได้ทั้งมุมมองภาพรวมและ
  รายละเอียดรายสถานี เพื่อให้ผู้ใช้ตรวจสอบข้อมูลเชิงลึกได้อย่างครบถ้วน นอกจากนี้ยังมี 
  dashboard สรุปผล ที่นำเสนอข้อมูลในรูปแบบกราฟและตัวชี้วัดสำคัญ 
  เพื่อช่วยให้ผู้ใช้สามารถเปรียบเทียบและวิเคราะห์ประสิทธิภาพของระบบขนส่งสาธารณะ
  ได้อย่างสะดวกและมีประสิทธิภาพ
\end{mypara}

\subsection{การจัดกลุ่มผู้ใช้ (User Grouping)}
\begin{mypara}
\indent โครงงานนี้มีการจัดกลุ่มผู้ใช้เป็น 2 กลุ่มหลัก ได้แก่
\begin{itemize}
    \item ผู้ใช้ที่ลงทะเบียน (Registered Users): กลุ่มของผู้ใช้ที่สร้างบัญชีและเข้าสู่ระบบแล้ว 
    สามารถเข้าถึงฟังก์ชันทั้งหมดของระบบ สามารถบันทึกการจำลองหรืออัพโหลดข้อมูลขึ้นไปบน Communityได้ 
    \item ผู้ใช้ที่ไม่ได้ลงทะเบียน (Guest Users): กลุ่มของผู้ใช้ทั่วไปที่ไม่ได้สร้างบัญชี 
    สามารถใช้ระบบจำลองได้แบบจำกัดฟังก์ชัน โดยทำการจำลองและดูผลลัพธ์ได้ชั่วคราว 
\end{itemize}
\end{mypara}
\section{ผังการทำงานของผู้ใช้ (User Flow)}
\begin{mypara}
    \indent โครงงานนี้มีการออกแบบผังการทำงานของผู้ใช้ สำหรับผู้ใช้ทั้ง 2 กลุ่ม เพื่อให้ผู้ใช้สามารถเข้าถึงฟังก์ชันต่าง ๆ ของระบบได้อย่างเหมาะสม 
    โดยมีการออกแบบดังนี้
\end{mypara}
  \subsection{ผู้ใช้ที่ไม่ได้ลงทะเบียน (Guest Users)}
  \begin{mypara}
    \indent ผังการทำงานของผู้ใช้ที่ไม่ได้ลงทะเบียน จะถูกจำกัดการเข้าถึงฟังก์ชันบางส่วนของระบบโดยมี
    การทำงานคือเมื่อผู้ใช้เข้ามาสู่หน้าแรก (Landing page) ซึ่งผู้ที่ไม่ได้ลงทะเบียนจะเข้าใช้
    ในโหมด Guest user โดยสามารถเลือกได้ว่าจะใช้ข้อมูลที่มีอยู่ใน Community 
    ซึ่งสามารถเลือกได้ระหว่างว่าจะนำข้อมูลกำหนดค่า (Configuration data) อย่างเดียว กับ เลือกใช้ร่วมกับ
    ข้อมูลสถานการณ์ (Scenario data) หรือเลือกที่จะ Setup ข้อมูลทั้งหมดเอง เมื่อเลือกข้อมูลที่จะใช้ได้แล้ว
    จะสามารถ เข้าสู่หน้าสำหรับการใส่ ข้อมูลสถานการณ์ (Scenario data) เพื่อกรอกข้อมูลสถานการณ์และปรับแต่งข้อมูลได้ เมื่อกรอกข้อมูลเสร็จ
    แล้วจะสามารถกดปุ่มดำเนินการเพื่อเริ่มการใช้แบบจําลอง (Simulation) หลังจากการจำลองเสร็จสิ้น
    ผู้ใช้จะสามารถเลือกดูผลลัพธ์ได้โดยเลือกดูได้ทั้งแบบ Dashboard สรุปผลข้อมูล หรือ Interacting Visualize
    ซึ่งจะเป็นข้อมูลทางสถิติที่เป็นผลลัพธ์ภาพรวมหรือข้อมูลรายละเอียดรายสถานี ดังภาพ \ref{fig:UserFlowUnregistered}
  \end{mypara}

  \subsection{ผู้ใช้ที่ลงทะเบียน (Registered Users)}
  
  \begin{mypara}
    \indent ผังการทำงานของผู้ใช้ที่ลงทะเบียน จะสามารถเข้าถึงฟังก์ชันทั้งหมดของระบบ โดยมี
    การทำงานคือ เมื่อผู้ใช้เข้ามาสู่หน้าแรก (Landing page) โดยผู้ใช้จะใช้โหมดลงทะเบียน สำหรับผู้ใช้ที่ลงทะเบียน
    แล้วจะสามารถเข้าสู่ระบบ (Login) ได้แต่หากไม่ได้ลงทะเบียน จะต้องทำการสมัครสมาชิก (Sign up) ก่อนโดย
    ในโหมด Login user โดยหลังจากเข้าสู่ระบบผู้ใช้จะเข้าสู่หน้า Work space ซึ่งเป็นการจัดการงานทั้งหมดของผู้ใช้
    โดยการการเริ่่มงานใหม่จะสามารถทำได้ 2 วิธี โดยวิธีที่ 1 คือการนำเข้าข้อมูลจาก Community ซึ่งสามารถเลือก
    ได้ระหว่างว่าจะนำข้อมูลกำหนดค่า (Configuration data) อย่างเดียวหรือเลือกใช้ร่วมกับ
    ข้อมูลสถานการณ์ (Scenario data) วิธีที่ 2 คือการสร้างงานใหม่ (New work) 
    โดยการสร้างงานจะต้องอ้างอิงกับ ข้อมูลการกำหนดค่า (Configuration data) ที่มีอยู่ หากต้องการ
    ข้อมูลกำหนดค่าใหม่สามารถสร้าง setup ใหม่เองได้ซึ่งข้อมูลกำหนดค่าที่สร้างขึ้นสามารถบันทึกไว้ใช้
    ในงานต่อไปได้และเลือกอัพโหลดข้อมูลขึ้นไปใน Community เพื่อให้ผู้อื่นสามารถนำไปใช้ได้
    เมื่อสร้างงานใหม่หรือเลือกงานที่มีอยู่แล้วได้ซึ่งงานจะสามารถบันทึกไว้ใช้และเลือกอัพโหลดข้อมูลขึ้นไปใน Community เพื่อให้ผู้อื่นสามารถนำไปใช้ได้
    แล้วหลังจากกำหนดงานที่ต้องการได้แล้วจะสามารถเข้าสู่หน้าสำหรับการใส่ 
    ข้อมูลสถานการณ์ (Scenario data) 
    เพื่อกรอกข้อมูลสถานการณ์และปรับแต่งข้อมูลได้ เมื่อกรอกข้อมูลเสร็จ
    แล้วจะสามารถกดปุ่มดำเนินการเพื่อเริ่มการใช้แบบจําลอง (Simulation) หลังจากการจำลองเสร็จสิ้น
    ผู้ใช้จะสามารถเลือกดูผลลัพธ์ได้โดยเลือกดูได้ทั้งแบบ Dashboard สรุปผลข้อมูล หรือ Interacting Visualize
    ซึ่งจะเป็นข้อมูลทางสถิติที่เป็นผลลัพธ์ภาพรวมหรือข้อมูลรายละเอียดรายสถานี 
    ดังภาพ \ref{fig:UserFlowRegistered} 
  \end{mypara}

  \begin{figure}
    \centering
    \includegraphics[width=\textwidth,height=0.95\textheight,keepaspectratio]{User_flow_-_guest.png}
    \caption{User flow  ของผู้ใช้ที่ไม่ได้ลงทะเบียน}
    \label{fig:UserFlowUnregistered}
  \end{figure}

    \begin{figure}
      \centering
      \includegraphics[width=\textwidth,height=0.95\textheight,keepaspectratio]{User_flow_-_login.png}
      \caption{User flow ของผู้ใช้ที่ลงทะเบียน}
      \label{fig:UserFlowRegistered}
    \end{figure}
      
\newpage
\section{โครงร่างหน้าจอ (Wireframe)}
\begin{mypara}
    \indent โครงงานนี้มีการออกแบบ โครงร่างหน้าจอ สำหรับหน้าต่างๆ ของระบบ 
    เพื่อใช้เป็นแนวทางในการพัฒนาและทดสอบระบบ โดยมีการออกแบบดังนี้

\subsection{โครงร่างหน้าจอผู้ใช้ที่ไม่ได้ลงทะเบียน (Guest Users / Unregistered Users)}
\begin{itemize}
    \item Step 1: ผู้ใช้จะเข้าสู่หน้า Landing page เพื่อเลือกระหว่างจะเข้าใช้ในโหมด Guest user หรือ Login user 
    ตามภาพ \ref{fig:WireframeHomepage}
      \begin{figure}[H]
        \centering
        \includegraphics[scale=0.35]
        {homepage.png}
        \caption{Wireframe ของหน้า Landing page}
        \label{fig:WireframeHomepage}
      \end{figure}

    \item Step 2: Guest user สามารถเลือกว่าจะใช้ข้อมูลที่มีอยู่ใน Community หรือเลือกที่จะ Setup ข้อมูลทั้งหมดเอง
    ตามภาพ \ref{fig:WireframeGuestDecision}
      \begin{figure}[H]
        \centering
        \includegraphics[scale=0.35]
        {guest_login.png}
        \caption{Wireframe ของหน้า Guest Decision}
        \label{fig:WireframeGuestDecision}
      \end{figure}

    \item Step 3: การใช้ข้อมูล Configuration data แบ่งเป็น 2 กรณี ดังนี้
    \begin{itemize}
        \item กรณีที่เลือกใช้จากข้อมูลที่มีอยู่ใน Community 
        ตามภาพ \ref{fig:WireframeCommunityConfigGuest} และ \ref{fig:WireframeCommunityConfigDetailGuest}
          \begin{figure}[H]
            \centering
            \includegraphics[scale=0.4]{conf_commu_guest.png}
            \caption{Wireframe ของหน้า Community Configuration data}
            \label{fig:WireframeCommunityConfigGuest}
          \end{figure}

          \begin{figure}[H]
            \centering
            \includegraphics[scale=0.4]{conf_commu_detail_guest.png}
            \caption{Wireframe ของหน้ารายละเอียด Configuration data ใน Community}
            \label{fig:WireframeCommunityConfigDetailGuest}
          \end{figure}

        \newpage
        \item กรณีที่ผู้ใช้เลือก Setup ข้อมูลทั้งหมดเอง จะเข้าสู่การ Setup Configuration ด้วยตนเอง
        ตามภาพ \ref{fig:WireframeSetupConfigGuest}
          \begin{figure}[H]
            \centering
            \includegraphics[scale=0.4]{conf_setup_guest.png}
            \caption{Wireframe ของหน้า Setup Configuration data สำหรับ Guest}
            \label{fig:WireframeSetupConfigGuest}
          \end{figure}
    \end{itemize}

    \item Step 4: หลังจาก Setup Configuration data ทั้งหมดเรียบร้อยจะเข้าสู่หน้า Input Scenario page 
    เพื่อกรอกข้อมูลสำหรับการจำลองบน Scenario ต่างๆ ตามภาพ \ref{fig:WireframeInputGuest} และ \ref{fig:WireframeInputRouteGuest}
      \begin{figure}[H]
        \centering
        \includegraphics[scale=0.4]{input_bus.png}
        \caption{Wireframe ของหน้า Input Scenario page (bus) }
        \label{fig:WireframeInputGuest}
      \end{figure}
      \begin{figure}[H]
        \centering
        \includegraphics[scale=0.4]{input_route.png}
        \caption{Wireframe ของหน้า Input Scenario page (route) }
        \label{fig:WireframeInputRouteGuest}
      \end{figure}

    \item Step 5: หลังจากผู้ใช้กดปุ่ม Run simulation แล้ว User จะสามารถเลือกดู  Dashboard สรุปผลข้อมูล 
    หรือ Interacting Visualize ตามภาพ \ref{fig:WireframeOutputGuest} และ \ref{fig:WireframeDashboardGuest}
      \begin{figure}[H]
        \centering
        \includegraphics[scale=0.4]{output_show.png}
        \caption{Wireframe ของหน้าแสดงผลลัพธ์หลังจาก Run Simulation}
        \label{fig:WireframeOutputGuest}
      \end{figure} 

      \begin{figure}[H]
        \centering
        \includegraphics[scale=0.4]{dashboard.png}
        \caption{Wireframe ของหน้า Dashboard }
        \label{fig:WireframeDashboardGuest}
      \end{figure}
    
\end{itemize} 


\subsection{โครงร่างหน้าจอของผู้ใช้ที่ลงทะเบียน (Registered Users)}
\begin{itemize}
    \item Step 1:  ผู้ใช้จะเข้าสู่หน้า Landing page เพื่อเลือกระหว่างจะเข้าใช้ในโหมด Guest user หรือ Login user
    ตามภาพ \ref{fig:WireframeHomepageLogin}
    \begin{figure}[H]
    \centering
    \includegraphics[scale=0.4]
    {homepage.png}
    \caption{Wireframe ของหน้า Landing page}
    \label{fig:WireframeHomepageLogin}
    \end{figure}

    \newpage
    \item Step 2: หลังจากเข้าสู่ระบบผู้ใช้จะเข้าสู่หน้า My work ซึ่งเป็นการจัดการงานทั้งหมดของผู้ใช้
    ตามภาพ \ref{fig:WireframeMyWork} และ \ref{fig:WireframeMyWorkPublish}
    \begin{itemize}
      \item ในหน้า My work ผู้ใช้สามารถเลือกใช้งานหรือ Publish งาน
        \begin{figure}[H]
          \centering
          \includegraphics[scale=0.4]{my_work.png} 
          \caption{Wireframe ของหน้า My work}
          \label{fig:WireframeMyWork}
        \end{figure}

        \begin{figure}[H]
          \centering
          \includegraphics[scale=0.6]{my_work_publish.png} 
          \caption{Wireframe ของหน้า My work ในกรณีที่ Publish งาน}
          \label{fig:WireframeMyWorkPublish}
        \end{figure}
        
      \newpage
      \item  ผู้ใช้สามารถสร้าง Work ใหม่ได้ด้วยการกดที่ปุ่ม New work โดยการสร้าง Work จะต้องอ้างอิงกับ Configuration data ที่มีอยู่
      ตามภาพ \ref{fig:WireframeNewWork}  
      \begin{figure}[H]
        \centering
        \includegraphics[scale=0.4]{new_work.png}
        \caption{Wireframe ของหน้า New work}
        \label{fig:WireframeNewWork}
      \end{figure}
    \end{itemize}

    \item Step 3: การใช้ข้อมูล Configuration data แบ่งเป็น 2 กรณี ดังนี้
    \begin{itemize}
        \item กรณีที่เลือกใช้จากข้อมูลที่มีอยู่ใน Community ตามภาพ \ref{fig:WireframeCommunityConfigLogin} และ \ref{fig:WireframeCommunityConfigDetailLogin}
          \begin{figure}[H]
            \centering
            \includegraphics[scale=0.4]{conf_commu.png}
            \caption{Wireframe ของหน้า Community Configuration data}
            \label{fig:WireframeCommunityConfigLogin}
          \end{figure}

          \begin{figure}[H]
            \centering
            \includegraphics[scale=0.4]{conf_commu_detail_reg.png}
            \caption{Wireframe ของหน้ารายละเอียด Configuration data ใน Community}
            \label{fig:WireframeCommunityConfigDetailLogin}
          \end{figure}

        \item ผู้ใช้สามารถสร้าง Configuration data ใหม่ได้ด้วยการกดที่ปุ่ม New Configuration และสามารถ Save Configuration 
        เพื่อใช้ในงานต่อไปได้ ตามภาพ \ref{fig:WireframeNewConfigLogin} และ \ref{fig:WireframeSetupConfigLogin}
          \begin{figure}[H]
            \centering
            \includegraphics[scale=0.4]{new_conf.png}
            \caption{Wireframe ของหน้า New Configuration data}
            \label{fig:WireframeNewConfigLogin}
          \end{figure}

          \begin{figure}[H]
            \centering
            \includegraphics[scale=0.4]{conf_setup_reg.png}
            \caption{Wireframe ของหน้า Setup Configuration data สำหรับผู้ใช้ที่ลงทะเบียน}
            \label{fig:WireframeSetupConfigLogin}
          \end{figure}

    \end{itemize}

    \item Step 4: หน้า Input Scenario page เพื่อกรอกข้อมูลสำหรับการจำลองบน Scenario ต่างๆ 
    ตามภาพ \ref{fig:WireframeInputLogin} และ \ref{fig:WireframeInputRouteLogin}
      \begin{figure}[H]
        \centering 
        \includegraphics[scale=0.4]{input_bus.png}
        \caption{Wireframe ของหน้า Input Scenario page (bus) }
        \label{fig:WireframeInputLogin}
      \end{figure}

      \begin{figure}[H]
        \centering
        \includegraphics[scale=0.4]{input_route.png}
        \caption{Wireframe ของหน้า Input Scenario page (route) }
        \label{fig:WireframeInputRouteLogin}
      \end{figure}

    \item Step 5: หลังจากผู้ใช้กดปุ่ม Run simulation แล้ว User จะสามารถเลือกดู  Dashboard สรุปผลข้อมูล 
    ตามภาพ \ref{fig:WireframeOutputLogin} และ \ref{fig:WireframeDashboardLogin}
      \begin{figure}[H]
        \centering
        \includegraphics[scale=0.4]{output_show.png}
        \caption{Wireframe ของหน้าแสดงผลลัพธ์หลังจาก Run simulation}
        \label{fig:WireframeOutputLogin}
      \end{figure}

      \begin{figure}[H]
        \centering
        \includegraphics[scale=0.4]{dashboard.png}
        \caption{Wireframe ของหน้า Dashboard }
        \label{fig:WireframeDashboardLogin}
      \end{figure}
    \end{itemize}

\end{mypara}

\section{รูปแบบไฟล์ข้อมูลนำเข้า (Supported Input File Formats)}
  \subsection{รูปแบบไฟล์ช่วงเวลาในการมาถึงของผู้โดยสาร (Passenger Arrival Time File Format)}
  \begin{mypara}
      \indent รูปแบบไฟล์ช่วงเวลาในการมาถึงของผู้โดยสาร ใช้สำหรับเก็บข้อมูลช่วงเวลาที่ผู้โดยสารแต่ละคนมาถึงที่สถานี
      โดยข้อมูลจะถูกจัดเก็บในไฟล์ Excel โดยมีสกุลไฟล์ \texttt{.xlsx} 
      ซึ่งมีโครงสร้างดังแสดงในภาพ \ref{fig:PassengerArrivalFileFormat}
      \begin{figure}[H]
        \centering
        \includegraphics[scale=0.5]{Passenger_Interarrival.png}
        \caption{รูปแบบไฟล์ช่วงเวลาในการมาถึงของผู้โดยสาร}
        \label{fig:PassengerArrivalFileFormat}
      \end{figure}
  \end{mypara}

  \newpage
  \subsection{รูปแบบไฟล์จำนวนผู้โดยสารที่ลงจากรถ (Passenger Alight File Format)}
  \begin{mypara}
      \indent รูปแบบไฟล์จำนวนผู้โดยสารที่ลงจากรถ ใช้สำหรับเก็บข้อมูลจำนวนผู้โดยสารที่ลงจากรถในแต่ละสถานี
      โดยข้อมูลจะถูกจัดเก็บในไฟล์ Excel โดยมีสกุลไฟล์ \texttt{.xlsx} 
      ซึ่งมีโครงสร้างดังแสดงในภาพ \ref{fig:PassengerAlightFileFormat}
      \begin{figure}[H]
        \centering
        \includegraphics[scale=0.5]{Passenger_alighting.png}
        \caption{รูปแบบไฟล์จำนวนผู้โดยสารที่ลงจากรถ}
        \label{fig:PassengerAlightFileFormat}
      \end{figure}
  \end{mypara}

  \subsection{รูปแบบไฟล์ตารางการเดินรถ (Bus Schedule File Format)}
  \begin{mypara}
      \indent รูปแบบไฟล์ตารางการเดินรถ ใช้สำหรับเก็บข้อมูลเวลาการออกรถของแต่ละสายบริการ 
      โดยข้อมูลจะถูกจัดเก็บในไฟล์ Excel โดยมีสกุลไฟล์ \texttt{.xlsx} 
      ซึ่งมีโครงสร้างดังแสดงในภาพ \ref{fig:BusScheduleFileFormat}
      \begin{figure}[H]
        \centering
        \includegraphics[scale=0.5]{bus_schedule.png}
        \caption{รูปแบบไฟล์ตารางการเดินรถ}
        \label{fig:BusScheduleFileFormat}
      \end{figure}
  \end{mypara}