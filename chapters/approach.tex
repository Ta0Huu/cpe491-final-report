\chapter{\ifproject%
\ifenglish Project Structure and Methodology\else โครงสร้างและขั้นตอนการทำงาน\fi
\else%
\ifenglish Project Structure\else โครงสร้างของโครงงาน\fi
\fi
}

\makeatletter

% \renewcommand\section{\@startsection {section}{1}{\z@}%
%                                    {13.5ex \@plus -1ex \@minus -.2ex}%
%                                    {2.3ex \@plus.2ex}%
%                                    {\normalfont\large\bfseries}}

\makeatother
%\vspace{2ex}
% \titleformat{\section}{\normalfont\bfseries}{\thesection}{1em}{}
% \titlespacing*{\section}{0pt}{10ex}{0pt}

\section{โครงสร้างโดยรวมของโครงงาน (Project Overview) }
\begin{mypara}
    \indent โครงงานนี้เป็นระบบสนับสนุนการตัดสินใจสำหรับการวางแผนระบบขนส่งสาธารณะ 
    โดยจะจัดทำเป็นเว็บแอพลิเคชั่นสำหรับการจำลอง และมีการสรุปผลลัพธ์ที่ได้จากการจำลองในรูปแบบทางสถิติ
    ซึ่งในการทำงานของระบบนี้จะมี 4 องค์ประกอบหลักดังนี้  
\end{mypara}


\begin{figure}[H]
    \centering
    \includegraphics[width=\textwidth,height=0.9\textheight,keepaspectratio]{overview.png}
    \caption{แผนภาพโดยรวมของโครงงาน}
    \label{fig:overview}
\end{figure}

\subsection{Input data (Scenario data)}
\begin{mypara}
  \indent เป็นข้อมูลที่ผู้ใช้สามารถปรับเปลี่ยนพารามิเตอร์ได้บ่อย ๆ เพื่อจำลองสถานการณ์ต่าง ๆ 
  สำหรับการเปรียบเทียบและวางแผน ประกอบด้วย เส้นทางการให้บริการของรถในแต่ละสายบริการ 
  ตารางเวลาการออกรถ ช่วงเวลาที่ต้องการดูผลการจําลอง ข้อมูลของรถซึ่งมี 
  ความจุผู้โดยสารระยะทางที่สามารถวิ่งได้สูงสุด ความเร็วในการวิ่ง
\end{mypara}
\subsection{Configuration data (Base data)}
\begin{mypara}
  \indent เป็นส่วนของข้อมูลที่ใช้เป็นพื้นฐานในการจำลองซึ่งจะไม่ค่อยมีการเปลี่ยนแปลงบ่อย 
  ได้แก่ ข้อมูลช่วงระยะห่างของเวลาที่ผู้โดยสารแต่ละคนมาถึงที่สถานี ข้อมูลของจำนวนผู้โดยสารที่ลงจากรถในแต่ละสถานี 
  และข้อมูลที่เกี่ยวข้องกับสถานีซึ่งจะเก็บในรูปแบบของ network model ซึ่งจะมี 
  node เป็นรายชื่อสถานี และ มี edge เป็นระยะทางละหว่างแต่ละสถานี
  \end{mypara}
\subsection{Simulation Engine}
\begin{mypara}
  \indent เป็นส่วนที่ทำหน้าที่ในการจำลองระบบขนส่งสาธารณะตามข้อมูลที่ได้รับจากส่วน input 
  และ configuration data ซึ่งจะใช้เทคนิคการจำลองแบบเหตุการณ์ไม่ต่อเนื่อง (discrete-event simulation) 
  ในการจำลองระบบขนส่งสาธารณะ 
\end{mypara}
\subsection{Output}
\begin{mypara}
  \indent เป็นส่วนที่แสดงผลลัพธ์ที่ได้จากการจำลอง โดยจะมีข้อมูล การรอเฉลี่ยของผู้โดยสาร อัตราการใช้งานของรถ 
  และอัตราการเกิดเหตุการณ์ที่รถบัสมีการใช้งานเกินเวลาหรือระยะทางที่กำหนด
  ซึ่งจะมีข้อมูลแสดงทั้งข้อมูลรวมทุกสถานี และข้อมูลรายละเอียดของแต่ละสถานีแยกกัน รวมถึงจะมี dashboard 
  สำหรับแสดงผลลัพธ์ในรูปแบบกราฟเพื่อให้ผู้ใช้สามารถวิเคราะห์ข้อมูลได้ง่ายขึ้น
\end{mypara}
\section{ ขั้นตอนการทำงาน (Process Overview)}
\begin{mypara}
    \indent โครงงานนี้ทำงานโดยนำข้อมูลจากผู้ใช้ (input) มาจำลองระบบขนส่งสาธารณะ 
    โดยอ้างอิงจาก configuration data ซึ่งประกอบด้วยข้อมูลหลายประเภท 
    โดยบางส่วนจะถูกนำไปสร้างเป็น distribution function เพื่อใช้ในการสุ่มพฤติกรรมของผู้โดยสาร 
    ขณะที่ข้อมูลอีกส่วนจะถูกนำมาใช้โดยตรง ใน simulation engine ซึ่งจะประมวลผลด้วย
    เทคนิคการจำลองแบบเหตุการณ์ไม่ต่อเนื่อง (discrete-event simulation) 
    เพื่อสร้างผลลัพธ์เชิงสถิติที่สะท้อนประสิทธิภาพของระบบขนส่งสาธารณะมาแสดงผลในส่วน output
\end{mypara}

\section{User Interface (UI)}
\begin{mypara}
    \indent โครงงานนี้มีการออกแบบเพื่อให้ผู้ใช้สามารถป้อนข้อมูลที่จำเป็นสำหรับ
    การจำลองระบบขนส่งสาธารณะได้อย่างสะดวกและไม่ซับซ้อน 
    โดยข้อมูลที่มีการเปลี่ยนแปลงบ่อย จะถูกจัดให้อยู่ในส่วนของ input 
    ซึ่งสามารถเข้าถึงและแก้ไขได้ง่ายโดยไม่ต้องผ่านขั้นตอนที่ซับซ้อน 
    ส่วนข้อมูลพื้นฐานของระบบขนส่งสาธารณะ จะถูกจัดเก็บแยกเป็น configuration data 
    เพื่อให้ง่ายต่อการนำมาใช้ซ้ำ และมี workspace สำหรับการจัดการข้อมูลที่ช่วยให้ผู้ใช้สามารถ 
    clone configuration data หรือ input จากผู้อื่นมาใช้งานและปรับแก้ได้สะดวก

  \indent ในส่วนของ output ได้ออกแบบให้สามารถแสดงผลได้ทั้งมุมมองภาพรวมและ
  รายละเอียดรายสถานี เพื่อให้ผู้ใช้ตรวจสอบข้อมูลเชิงลึกได้อย่างครบถ้วน นอกจากนี้ยังมี 
  dashboard สรุปผล ที่นำเสนอข้อมูลในรูปแบบกราฟและตัวชี้วัดสำคัญ 
  เพื่อช่วยให้ผู้ใช้สามารถเปรียบเทียบและวิเคราะห์ประสิทธิภาพของระบบขนส่งสาธารณะ
  ได้อย่างสะดวกและมีประสิทธิภาพ
\end{mypara}

\subsection{การจัดกลุ่มผู้ใช้ (User Grouping)}
\begin{mypara}
\indent โครงงานนี้มีการจัดกลุ่มผู้ใช้เป็น 2 กลุ่มหลัก ได้แก่
\begin{itemize}
    \item ผู้ใช้ที่ลงทะเบียน (Registered Users): กลุ่มนี้เป็นผู้ใช้ที่สร้างบัญชีและเข้าสู่ระบบเรียบร้อยแล้ว 
    สามารถเข้าถึงฟังก์ชันทั้งหมดของระบบ สามารถบันทึกการจำลองหรือ upload ข้อมูลขึ้นไปบน workspace ได้ 
    \item ผู้ใช้ที่ไม่ได้ลงทะเบียน (Guest Users / Unregistered Users): กลุ่มนี้เป็นผู้ใช้ทั่วไปที่ไม่ได้สร้างบัญชี 
    สามารถใช้ระบบจำลองได้แบบจำกัดฟังก์ชัน โดยทำการจำลองและดูผลลัพธ์ได้ชั่วคราว 
    แต่ไม่สามารถบันทึกข้อมูลได้
\end{itemize}
\end{mypara}
\section{User Flow}

\begin{mypara}

\begin{itemize}
    \item ผู้ใช้ที่ไม่ได้ลงทะเบียน (Guest Users / Unregistered Users)
    \begin{figure}[H]
    \centering
    \includegraphics[width=\textwidth,height=0.6\textheight,keepaspectratio]{User_flow_-_guest.png}
    \caption{user flow  ของผู้ใช้ที่ไม่ได้ลงทะเบียน}
    \label{fig:UserFlowUnregistered}
    \end{figure}
    \newpage
    \item ผู้ใช้ที่ลงทะเบียน (Registered Users)
    \begin{figure}[H]
    \centering
    \includegraphics[width=\textwidth,height=0.9\textheight,keepaspectratio]{User_flow_-_login.png}
    \caption{user flow ของผู้ใช้ที่ลงทะเบียน}
    \label{fig:UserFlowRegistered}
    \end{figure}
    

\end{itemize}
\end{mypara}
\newpage
\section{Wireframe}
\begin{mypara}
    \indent โครงงานนี้มีการออกแบบ wireframe สำหรับหน้าต่างๆ ของระบบ 
    เพื่อใช้เป็นแนวทางในการพัฒนาและทดสอบระบบ โดย wireframe ที่ออกแบบมีดังนี้

\subsection{Wireframe ผู้ใช้ที่ไม่ได้ลงทะเบียน (Guest Users / Unregistered Users)}
\begin{itemize}
    \item Step 1: ผู้ใช้เข้าหน้า Landing page เพื่อเลือกระหว่าง guest และ Login
      \begin{figure}[H]
        \centering
        \includegraphics[scale=0.4]
        {homepage.png}
        \caption{wireframe ของหน้า Landing page}
        \label{fig:WireframeHomepage}
      \end{figure}

    \item Step 2: Guest เลือกว่าจะใช้ข้อมูลที่อยู่ใน community หรือ Setup configuration เอง
      \begin{figure}[H]
        \centering
        \includegraphics[scale=0.4]
        {guest_login.png}
        \caption{wireframe ของหน้า Guest decision}
        \label{fig:WireframeGuestDecision}
      \end{figure}

    \item Step 3: การใช้ข้อมูล Configuration Data
    \begin{itemize}
        \item เลือกใช้จากข้อมูลที่อยู่ใน community
          \begin{figure}[H]
            \centering
            \includegraphics[scale=0.4]{conf_commu_guest.png}
            \caption{wireframe ของหน้า Community configuration data}
            \label{fig:WireframeCommunityConfigGuest}
          \end{figure}

          \begin{figure}[H]
            \centering
            \includegraphics[scale=0.4]{conf_commu_detail_guest.png}
            \caption{wireframe ของหน้ารายละเอียด configuration data ใน community}
            \label{fig:WireframeCommunityConfigDetailGuest}
          \end{figure}

        \newpage
        \item ผู้ใช้เลือก Setup configuration ด้วยตนเอง
          \begin{figure}[H]
            \centering
            \includegraphics[scale=0.4]{conf_setup_guest.png}
            \caption{wireframe ของหน้า Setup configuration data สำหรับ guest}
            \label{fig:WireframeSetupConfigGuest}
          \end{figure}
    \end{itemize}

    \item Step 4: หน้า input page เพื่อกรอกข้อมูล input ต่างๆ
      \begin{figure}[H]
        \centering
        \includegraphics[scale=0.4]{input_bus.png}
        \caption{wireframe ของหน้า input page (bus) }
        \label{fig:WireframeInputGuest}
      \end{figure}
      \begin{figure}[H]
        \centering
        \includegraphics[scale=0.4]{input_route.png}
        \caption{wireframe ของหน้า input page (route) }
        \label{fig:WireframeInputRouteGuest}
      \end{figure}

    \item Step 5:หลัง Run simulation แล้ว user จะสามารถเลือกดู  Dashboard สรุปผลข้อมูล 
    หรือ Interacting Visualize
      \begin{figure}[H]
        \centering
        \includegraphics[scale=0.4]{output_show.png}
        \caption{wireframe ของหน้าแสดงผลลัพธ์หลังจาก run simulation}
        \label{fig:WireframeOutputGuest}
      \end{figure} 

      \begin{figure}[H]
        \centering
        \includegraphics[scale=0.4]{dashboard.png}
        \caption{wireframe ของหน้า Dashboard }
        \label{fig:WireframeDashboardGuest}
      \end{figure}
    
\end{itemize} 


\subsection{wireframe ของผู้ใช้ที่ลงทะเบียน (Registered Users)}
\begin{itemize}
    \item Step 1: ผู้ใช้เข้าหน้า Landing page เพื่อเลือกระหว่าง guest และ Login
    \begin{figure}[H]
    \centering
    \includegraphics[scale=0.4]
    {homepage.png}
    \caption{wireframe ของหน้า Landing page}
    \label{fig:WireframeHomepageLogin}
    \end{figure}
    \item Step 2: My work จัดการงานของผู้ใช้
    \begin{itemize}
      \item My work เลือกใช้งานหรือ publish งาน
        \begin{figure}[H]
          \centering
          \includegraphics[scale=0.4]{my_work.png} 
          \caption{wireframe ของหน้า My work}
          \label{fig:WireframeMyWork}
        \end{figure}

        \begin{figure}[H]
          \centering
          \includegraphics[scale=0.6]{my_work_publish.png} 
          \caption{wireframe ของหน้า My work ในกรณีที่ publish งาน}
          \label{fig:WireframeMyWorkPublish}
        \end{figure}

      \item  New work ซึ่งสามารถสร้าง work จาก config ที่มีอยู่
      \begin{figure}[H]
      \centering
      \includegraphics[scale=0.4]{new_work.png}
      \caption{wireframe ของหน้า New work}
      \label{fig:WireframeNewWork}
      \end{figure}
    \end{itemize}
    \item Step 3: การใช้ข้อมูล Configuration Data
    \begin{itemize}
        \item ใช้จากข้อมูลที่อยู่ใน community
        \begin{figure}[H]
        \centering
        \includegraphics[scale=0.4]{conf_commu.png}
        \caption{wireframe ของหน้า Community configuration data}
        \label{fig:WireframeCommunityConfigLogin}
        \end{figure}

        \begin{figure}[H]
        \centering
        \includegraphics[scale=0.4]{conf_commu_detail_reg.png}
        \caption{wireframe ของหน้ารายละเอียด configuration data ใน community}
        \label{fig:WireframeCommunityConfigDetailLogin}
        \end{figure}

        \item New configuration สร้าง config ใหม่และสามารถ save config ได้
        \begin{figure}[H]
        \centering
        \includegraphics[scale=0.4]{new_conf.png}
        \caption{wireframe ของหน้า New configuration data}
        \label{fig:WireframeNewConfigLogin}
        \end{figure}
        \begin{figure}[H]
        \centering
        \includegraphics[scale=0.4]{conf_setup_reg.png}
        \caption{wireframe ของหน้า Setup configuration data สำหรับผู้ใช้ที่ลงทะเบียน}
        \label{fig:WireframeSetupConfigLogin}
        \end{figure}

    \end{itemize}
    \item Step 4: หน้า input page เพื่อกรอกข้อมูล input ต่างๆ
    \begin{figure}[H]
    \centering 
    \includegraphics[scale=0.4]{input_bus.png}
    \caption{wireframe ของหน้า input page (bus) }
    \label{fig:WireframeInputLogin}
    \end{figure}
    \begin{figure}[H]
    \centering
    \includegraphics[scale=0.4]{input_route.png}
    \caption{wireframe ของหน้า input page (route) }
    \label{fig:WireframeInputRouteLogin}
    \end{figure}
    \item Step 5:หลัง Run simulation แล้ว user จะสามารถเลือกดู  Dashboard สรุปผลข้อมูล
    หรือ Interacting Visualize
    \begin{figure}[H]
    \centering
    \includegraphics[scale=0.4]{output_show.png}
    \caption{wireframe ของหน้าแสดงผลลัพธ์หลังจาก run simulation}
    \label{fig:WireframeOutputLogin}
    \end{figure}
    \begin{figure}[H]
    \centering
    \includegraphics[scale=0.4]{dashboard.png}
    \caption{wireframe ของหน้า Dashboard }
    \label{fig:WireframeDashboardLogin}
    \end{figure}
    \end{itemize}

\end{mypara}